\documentclass{article}
\usepackage[utf8]{inputenc}
\usepackage[leqno]{mathtools}
\usepackage{amsthm}
\usepackage{amsfonts}
\usepackage[margin=0.75in]{geometry}

\DeclareMathOperator{\lcm}{lcm}

\newtheorem{thm}{Theorem}[section]
\newtheorem{ques}[thm]{Question}
\newtheorem{ex}[thm]{Exercise}
\newtheorem{cor}[thm]{Corollary}
\newtheorem{lem}{Lemma}[thm]
\newtheorem{PC}{Paul's Conjecture}

\numberwithin{equation}{thm}

\providecommand{\gmod}[1]{\; (\bmod \; #1)}

\usepackage{subfiles}

\title{Number Theory Notebook}
\author{Paul Schulze}
\date{January 22, 2021}

\begin{document}

\maketitle



\section{Chapter 1}

\subsection*{Divisibility and congruence}

\subfile{Chapter_1/Divisibility_and_Congruence}


\subsection*{The Division Algorithm}

\subfile{Chapter_1/The_Division_Algorithm}


\subsection*{Greatest common divisors and linear Diophantine equations}

\subfile{Chapter_1/GCD_and_linear_Diophantine}



\pagebreak



\section{Chapter 2}

\subsection*{Fundamental Theorem of Arithmetic}

\subfile{Chapter_2/Fundamental_Theorem_of_Arithmetic}


\subsection*{Applications of the Fundamental Theorem of Arithmetic}

\subfile{Chapter_2/Applications_of_the_FTA}


\subsection*{The infinitude of primes}

\subfile{Chapter_2/The_infinitude_of_primes}


\subsection*{Primes of special form}

\subfile{Chapter_2/Primes_of_special_form}


\subsection*{The distribution of primes}

\subfile{Chapter_2/The_distribution_of_primes.tex}



\pagebreak



\section{Chapter 3}

\subsection*{Powers and Polynomials modulo $n$}


\begin{ex} \label{3.1}
  Show that $41$ divides $2^{20} - 1$ by following these steps. Explain why each step is true.
\end{ex}

1. $2^5 \equiv -9 \gmod{41}$ because $41 | 41 = 32 - (-9)$

2. $(2^5)^4 \equiv (-9)^4 \gmod{41}$ because \ref{1.14}

3. $2^{20} \equiv 81^2 \gmod{41}$ because $(2^5)^4 = 2^{5 \cdot 4} = 2^{20}$ and $(-9)^4 = ((-9)^2)^2 = 81^2$

4. $2^{20} - 1 \equiv 0 \gmod{41}$ because $81 \equiv -1 \gmod{41}$ so $81^2 \equiv 1 \gmod{41}$ so $2^{20} \equiv 1 \gmod{41}$



\begin{ques} \label{3.2}
  In your head, can you find the natural number $k$, $0 \leq k \leq 11$, such that $k \equiv 37^{453} \gmod{12}$?
\end{ques}

$1$



\begin{ques} \label{3.3}
  In your head or using paper and pencil, but no calculator, can you find the natural number $k$, $0 \leq k \leq 6$, such that $2^{50} \equiv k \gmod{7}$?
\end{ques}

$4$



\begin{ques} \label{3.4}
  Using paper and pencil, but no calculator, can you find the natural number $k$, $0 \leq k \leq 11$, such that $39^{453} \equiv k \gmod{12}$?
\end{ques}

$3$



\begin{ex} \label{3.5}
  Show that $39$ divides $17^{48} - 5^{24}$
\end{ex}

So open up python and... no? Fine.

Notice (via calculator) that $17^2 \equiv 16 \gmod{39}$. Then, $17^3 \equiv 16 \cdot 17 \equiv -1 \gmod{39}$. From there we obtain $17^4 \equiv -17$ and $17^5 \equiv -16$ before arriving at $17^6 \equiv 1$. This cycle then repeats: we notice that $48$ is a multiple of $6$, so $17^{48} \equiv 1 \gmod{39}$.

Similarly, $5^4 \equiv 1 \gmod{39}$ so we find $5^{24} \equiv 1$. Then we know $17^{48} \equiv 5^{24} \gmod{39}$, so definitionally $39 | (17^{48} - 5^{24})$.



\begin{ques} \label{3.6}
  Let $a$, $n$, and $r$ be natural numbers. Describe how to find the number $k$ ($0 \leq k \leq n-1$) such that $k \equiv a^{r} \mod n$ subject to the restraint that you never multiply numbers larger than $n$ and that you only have to do about $\log_2 (r)$ such multiplications.
\end{ques}

Get a computer to do it.

Failing that, or if you're the poor soul who has to program the computer, you start off by essentially making a library of powers of $a$ in their ``simplest form'' modulo $n$. So you start with $a \equiv x_0 \gmod n$ (where in most cases if $a < n$ we find $x_0 = a$, but then you square $a$ and then pair down the result to something between $0$ and $n$ (probably using the division algorithm) to get $a^2 \equiv x_1$, then square that to get $a^4 \equiv x_2$, $a^8 \equiv x_3$, etc. etc. until you've build up to $x_{\lfloor \log_2 (r) \rfloor}$. Then add together the relevant terms from largest to smallest (e.g. $a^{39} \equiv a^{32} + a^4 + a^2 + a \equiv x_5 + x_2 + x_1 + x_0$).



\end{document}
