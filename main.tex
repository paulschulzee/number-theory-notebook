\documentclass{article}
\usepackage[utf8]{inputenc}
\usepackage[leqno]{mathtools}
\usepackage{amsthm}
\usepackage{amsfonts}
\usepackage[margin=0.75in]{geometry}

\DeclareMathOperator{\lcm}{lcm}

\newtheorem{thm}{Theorem}[section]
\newtheorem{ques}[thm]{Question}
\newtheorem{ex}[thm]{Exercise}
\newtheorem{cor}[thm]{Corollary}
\newtheorem{lem}{Lemma}[thm]
\newtheorem{PC}{Paul's Conjecture}

\numberwithin{equation}{thm}

\providecommand{\gmod}[1]{\; (\bmod \; #1)}

\usepackage{subfiles}

\title{Number Theory Notebook}
\author{Paul Schulze}
\date{January 22, 2021}

\begin{document}

\maketitle



\section{Chapter 1}

\subsection*{Divisibility and congruence}

\subfile{Chapter_1/Divisibility_and_Congruence}


\subsection*{The Division Algorithm}

\subfile{Chapter_1/The_Division_Algorithm}


\subsection*{Greatest common divisors and linear Diophantine equations}

\subfile{Chapter_1/GCD_and_linear_Diophantine}



\pagebreak



\section{Chapter 2}

\subsection*{Fundamental Theorem of Arithmetic}

\subfile{Chapter_2/Fundamental_Theorem_of_Arithmetic}


\subsection*{Applications of the Fundamental Theorem of Arithmetic}

\begin{thm} \label{2.12}
  Let $a$ and $b$ be natural numbers greater than $1$ and let $p_1^{r_1} p_2^{r_2} \cdots p_m^{r_m}$ be the unique prime factorization of $a$ and let $q_1^{t_1} q_2^{t_2} \cdots q_s^{t_s}$ be the unique prime factorization of $b$. Then $a | b$ if and only if for all $i \leq m$ there exists a $j \leq s$ such that $p_i = q_j$ and $r_i \leq t_j$.
\end{thm}

\begin{proof}
  Woo boy. Let's start by showing $a | b$ implies... all of that.

  We know that $a | b$ means $ak = b$. This means that $p_1^{r_1} \cdots p_m^{r_m} k = q_1^{t_1} \cdots q_s^{t_2}$. Since $k$ is an integer (and a natural number, given both $a$ and $b$ are natural) we know that it has its own unique prime factorization. Thus, the prime factorization of $a$ times the prime factorization of $k$ must be equal to the prime factorization of $b$ (since $ak = b$ and prime factorizations are unique).

  When we multiply the prime factorization of $a$ by that $k$, we cannot remove any of the terms $p_1 \ldots p_m$, nor can we reduce any of the exponents $r_1 \ldots r_m$, since the prime factorization of $k$ will not contain the multiplicative inverse of any prime. Thus, in order for our product to be the prime factorization of $b$, all of the primes $p_1 \ldots p_m$ must also be included in the prime factorization of $b$, and all of the exponents $r_1 \ldots r_m$ must be less than or equal to the corresponding exponents in the prime factorization of $b$. \\[0ex]

  Now the other direction. If we know that for all $i \leq m$ there exists a $j \leq s$ such that $p_1 = q_j$ and $r_1 \leq t_j$, then we can rewrite $b$ as $(p_1^{r_1} \cdots p_m^{r_m}) \cdot (q_1^{t_1'} \cdots q_s^{t_s'})$, where $t_1' \ldots t_s'$ are the exponents on the relative prime modified to accomodate ``moving'' $p_1^{r_1}$ through $p_m^{r_m}$ to the front of the product (these exponents notably may be $0$).
  Since $q_1^{t_1'} \cdots q_s^{t_s'}$ is an integer (call it $k$) and $p_1^{r_1} \cdots p_m^{r_m}$, we've shown that $b = ak$, or in other words $a | b$.
\end{proof}



\begin{thm} \label{2.13}
  If $a$ and $b$ are natural numbers and $a^2 | b^2$, then $a | b$.
\end{thm}

\begin{proof}
  Let $a = p_1^{r_1} \cdots p_m^{r_m}$ and $b = q_1^{t_1} \cdots q_s^{t_s}$ be the unique prime factorizations of these numbers.

  Notice that $a^2 = p_1^{2r_1} \cdots p_m^{2r_m}$ and $b^2 = q_1^{2t_1} \cdots q_s^{2t_s}$, and that these are the prime factorizations of these numbers.

  By \ref{2.12}, we find $a^2 | b^2$ implies that for all $i \leq m$ there exists a $j \leq s$ such that $p_i = q_j$ and $2r_i \leq 2t_j$. This implies that $r_i \leq t_j$, so we conclude that for all $i \leq m$ there exists a $j \leq s$ such that $p_i = q_j$ and $r_i \leq t_j$. By \ref{2.12}, this means $a | b$.
\end{proof}



\begin{ex} \label{2.14}
  Find $(3^14 \cdot 7^22 \cdot 11^5 \cdot 17^3, 5^2 \cdot 11^4 \cdot 13^8 \cdot 17)$
\end{ex}

$11^4 \cdot 17$



\begin{ex} \label{2.15}
  Find $\lcm (3^14 \cdot 7^22 \cdot 11^5 \cdot 17^3, 5^2 \cdot 11^4 \cdot 13^8 \cdot 17)$
\end{ex}

$3^14 \cdot 5^2 \cdot 7^22 \cdot 11^5 \cdot 13^8 \cdot 17^3$



\begin{ex} \label{2.16}
  Make a conjecture that generalizes the ideas you used to solve the two previous exercises.
\end{ex}

\begin{PC} \label{PC 2.16}
  Let the primes be denoted $p_i$, where $p_1 = 2$, $p_2 = 3$, $p_3 = 5$, $p_4 = 7$, $p_5 = 11$, etc.

  Let $a = p_1^{r_1} p_2^{r_2} \cdots$ and $b = p_1^{t_1} p_2^{t_2} \cdots$ be the prime factorizations of natural numbers $a$ and $b$, where $r_i$ and $t_j$ can be $0$ to indicate the absence of a prime. Then

  $$\gcd (a, b) = p_1^{\min(r_1, t_1)} p_2^{\min(r_2, t_2)} \cdots = \prod_{i \in \mathbb{N}} p_i^{\min(r_i, t_i)} \mbox{ and}$$
  $$\lcm (a, b) = p_1^{\max(r_1, t_1)} p_2^{\max(r_2, t_2)} \cdots = \prod_{i \in \mathbb{N}} p_i^{\max(r_i, t_i)}.$$
\end{PC}



\begin{ques} \label{2.17}
  Do you think this method is always better, always worse, or sometimes better and sometimes worse than using the Euclidean Algorithm to find $(a,b)$? Why?
\end{ques}

For large numbers, for humans, this method is probably better because you can use divisibility rules to easily start prime factorizing the number and make it smaller and easier to work with (although, for numbers that are the products of only large primes like $29 \cdot 31$ or something this might be painful).

For computers the Euclidean Algorithm is almost certainly better, because you can easily find the quotient/remainder through repeated addition and it's very fast.



\pagebreak



\begin{thm} \label{2.18}
  Given $n+1$ natural numbers, say $a_1, a_2, \ldots, a_{n+1}$, all less than or equal to $2n$, then there exists a pair, say $a_i$ and $a_j$ with $i \neq j$, such that $a_i | a_j$.
\end{thm}

\begin{proof}
  Credit to Sam.

  We'll use the pigeonhole principle. We'll form sets $S_1$ through $S_{2n-1}$ for all \emph{odd} indices, with each set $S_t = \{t \cdot 2^n \mid n \in \mathbb{N} \cup \{ 0\} \}$.

  Notice that if any two of our numbers $a_i$ and $a_j$ fall into the same $S_t$, then $a_i = t \cdot 2^{n}$ and $a_j = t \cdot 2^{m}$. Assuming WLOG that $n \leq m$, we find $a_j = a_i \cdot 2^{m-n}$, which since $2^{m-n}$ is an integer (as $m - n \geq 0$) shows $a_i | a_j$. Thus, we only need show that two of our $a$'s fall into the same set.

  Notice then that every $a_i$ will fall into at least (in fact, exactly) one $S_t$. Take $a_i$'s prime factorization, let it be $p_1^{r_1} \cdots p_m^{r_m}$.

  Then if $2 \neq p_v$ for any $v \leq n$, since $a_i$ is odd we find $a_i = a_i \cdot 2^{0}$ and thus $a_i \in S_{a_i}$.

  If $2 = p_v$ for some $v \leq n$, then we notive $a_i = 2^{r_v} \cdot (p_1^{r_1} \cdots p_{v-1}^{r_{v-1}} p_{v+1}^{r_{v+1}} \cdots p_m^{r_m})$. Let's define that second term as $k$. Since all of the $p$'s are unique, we know that $k$ is odd, and thus $a_i = S_{k}$.

  Since there are only sets $S$ for each of the odd numbers between $1$ and $2n$, there are exactly $n$ sets. Since there are $n+1$ numbers in our set of $a$'s, by the pigeonhole principle, we know there must be $a_i, a_j$ such that $i \neq j$, $a_i \in S_t$, and $a_j \in S_t$ for some $t$. As we showed earlier, this implies that either $a_i | a_j$ or vice versa, completing our proof.
\end{proof}



\begin{thm} \label{2.19}
  There do not exist natural numbers $m$ and $n$ such that $7m^2 = n^2$.
\end{thm}

\begin{proof}
  In the prime factorization of $n^2$ the exponent on $7$ must be even, as it is double the exponent on $7$ in the prime factoization of $n$.

  In the prime factorization of $7m^2$, the exponent on $7$ must be odd, as it is double the exponent on $7$ in the prime factorization of $m$ \emph{plus one} (for multiplying by 7).

  Since prime factorizations are unique, $7m^2$ and $n^2$ having different exponents (for the same number cannot be both even \emph{and} odd) implies they are different numbers and thus not equal.
\end{proof}



\begin{thm} \label{2.20}
  There do not exist natural numbers $m$ and $n$ such that $24m^3 = n^3$.
\end{thm}

\begin{proof}
  Notice $24m^3 = 2^3 \cdot 3 \cdot m^3$.

  Then, apply the logic above to the exponent on $3$ in the prime factorization of these two numbers; it must be a multiple of $3$ in $n^3$, and yet it must be one \emph{more} than a multiple of $3$ in $24m^3 = 2^3 \cdot 3 \cdot m^3$. Since the same number cannot be both (as $0 \not \equiv 1 \gmod 3$), the prime factorizations of $n^3$ and $24m^3$ are not the same and thus the numbers cannot be equal.
\end{proof}



\begin{ex} \label{2.21}
  Show that $\sqrt{7}$ is irrational. That is, there do not exist natural numbers $n$ and $m$ such that $\sqrt{7} = \frac{n}{m}$.
\end{ex}

\begin{proof}
  If there were such numbers $n$ and $m$, then we would find $\sqrt{7} \cdot m = n$, implying $7m^2 = n^2$, a contradiction with \ref{2.19}. Thus, no such numbers exist.
\end{proof}



\begin{ex} \label{2.22}
  Show that $\sqrt{12}$ is irrational.
\end{ex}

If $\sqrt{12} = \frac{a}{b}$ for integers $a, b$, then $12b^2 = a^2$, which is impossible due to the smame logic we used in \ref{2.20}.



\begin{ex} \label{2.23}
  Show that $7^{\frac{1}{3}}$ is irrational.
\end{ex}

If $7^{\frac 1 3} = \frac{a}{b}$, then $7b^3 = a^3$. This is impossible for integers $a$ and $b$, as the prime factorization of $7b^3$ has an exponent on $7$ that is one \emph{greater} than a multiple of $3$, while the prime factoization of $a^3$ has an exponnet on $7$ that \emph{is} a multiple of $3$.



\begin{ques} \label{2.24}
  What other numbers can you show to be irrational? Make and prove the most general conjecture you can.
\end{ques}

\begin{PC} \label{PC 2.24}
  Let $w$ be an integer, with $w = p_1^{r_1} \cdot p_2^{r_2} \cdots p_t^{r_t}$ its prime factorization. $w^{\frac n m}$ (where $n$ and $m$ are integers) is irrational if for any one prime $p_i$ with $1 \leq i \leq t$ it is the case that $m \not | \; (n \cdot r_i)$.
\end{PC}

\begin{proof}
  Say it is the case there exists a $p_i$ such that $m \not | \; (n \cdot r_i)$. We then find $p_i^{r_i} | w_i$, so let's invoke $k$ such that $w = kp_i^{r_i}$. Notice $p_i \not | \; k$ due to the fact that $k = \frac{w}{p_i^{r_i}}$ which has no $p_i$'s in its prime factorization. We then find $w ^ {\frac n m} = k^{\frac n m} \cdot p_i ^ {\frac{nr_i}{m}}$.

  Say $w^{\frac n m} = k^{\frac n m} \cdot p_i^{\frac{nr_i}{m}} = \frac{a}{b}$. To show $w^{\frac n m}$ is irrational, we will asssume $a$ and $b$ are both integers and . Rearranging the equation, we find $k^{\frac n m} \cdot p_i^{\frac{nr_i}{m}} \cdot b = a$, and then raising both sides to the $m$th powerwe find $k^n \cdot p_i^{nr_i} \cdot b^m = a^m$.
  Now both sides of this equation are integers, which means we can compare their prime factorizations. Specifically, we are going to look at the exponent on $p_i$ in these prime factorizations. $k$ has no impact on this exponent (since $p_i \not | \; k$). $b^m$ provides some multiple of $m$ to this exponent, while $p_i$ provides $nr_i$; in other words, the exponent on $p_i$ on the left side is of the form $\alpha m + nr_i$ for some integer $\alpha$. On the left, since we only have $a^m$, we have an exponent of the form $\beta m$ for some integer $\beta$. Since these two are equal, we find $\alpha m + nr_i = \beta m$, which tells us $m | (\alpha m + nr_i)$, which then since $m | \alpha m$ we cite \ref{1.2} to find $m | nr_i$, a contradiction.

  Thus, $a$ and $b$ cannnot both be integers, and thus $w^{\frac n m}$ is irrational.
\end{proof}



\begin{thm} \label{2.25}
  Let $a$, $b$, and $n$ be integers. If $a|n$, $b|n$, and $(a, b) = 1$, then $ab | n$.
\end{thm}

\begin{proof}
  Let $n = p_1^{r_1} \cdots p_m^{r_m}$ be the prime factorization of $n$. Then, by \ref{2.12}, we can write $a = p_1^{\alpha_1} \cdots p_m^{\alpha_m}$ and $b = p_1^{\beta_1} \cdots p_m^{\beta_m}$, where for all $i$ with $1 \leq i \leq m$ we know $0 \leq \alpha_i, \beta_i \leq r_i$.

  Say $ab \not | \; n$. Since $ab = p_1^{\alpha_1 + \beta_1} \cdots p_m^{\alpha_m + \beta_m}$, we invoke \ref{2.12} to find there must be some $j$ such that $1 \leq j \leq m$ and $\alpha_j + \beta_j > r_j$ (since all of the primes in the prime factorization of $ab$ are represented in $p_1$ through $p_m$). However, since $\alpha_j \leq r_j$, this implies $\beta_j \geq 1$,
  and similarly that $\alpha_j \geq 1$. This means that since $p_j^{\alpha_j} | a$, we know $p_j | a$, and similarly that $p_j | b$ making $p_j$ a common divisor of $a$ and $b$. Since $p_j$ is prime, $p_j > 1$, and since $(a,b) = 1$, we find that $p_j$ is a common divisor of $a$ and $b$ greater than their greatest common divisor, a contradction. Thus, our assumption that $ab \not | \; n$ is false, and we find $ab | n$.
\end{proof}



\begin{thm} \label{2.26}
  Let $p$ be a prime and let $a$ be an integer. Then $p$ does not divide $a$ if and only if $(a, p) = 1$.
\end{thm}

\begin{proof}
  That $p$ does not divide $a$ if $(a, p) = 1$ is trivial, since if it did it would be a common divisor (as $p | p$) that is greater than $1$ (since $p$ is prime) which is impossible since $(a, p) = 1$.

  All that is left is to show that $p \not | \; a$ implies that $(a, p) = 1$. Since $p$ is prime, its only (natural) divisors are $1$ and $p$. Thus, these are the only candidates for $(a, p)$. Since $p \not | \; a$, though, $p$ is not a common divisor, and thus the only possiblity for $(a, p)$ is $1$.
\end{proof}



\begin{thm} \label{2.27}
  Let $p$ be a prime and let $a$ and $b$ be integers. If $p | ab$, then $p | a$ or $p | b$.
\end{thm}

\begin{proof}
  By \ref{2.12}, we know $p$ has to show up in the prime factorization of $ab$. Since the prime factorization of $ab$ only includes primes found in either the factorization of either $a$ or $b$ (as it can be obtained by replacing $a$ and $b$ with their prime factorizations and moving the terms around), this means $p$ must show up in the prime factorization of $a$ or $b$, and thus by \ref{2.12} either $p | a$ or $p | b$.
\end{proof}



\begin{thm} \label{2.28}
  Let $a$, $b$, and $c$ be integers. If $(b, c) = 1$, then $(a, bc) = (a,b) \cdot (a,c)$.
\end{thm}

\begin{proof}
  We invoke \ref{1.40} twice to find integers $x_1, x_2, y_1, y_2$ such that $ax_1 + bx_2 = (a,b)$ and $ay_1 + cy_2 = (a,c)$. Then, we multiply the two equations together to get $(ax_1 + bx_2) \cdot (ay_1 + cy_2) = (a,b) \cdot (a,c)$. With some rearrangement we find $a^2x_1y_1 + abx_2y_1 + acx_1y_2 + bcx_2y_2 = (a,b) \cdot (a,c)$, which with distributivity we find means $a(ax_1y_1 + bx_2y_1 + cx_1y_2) + bc(x_2y_2) = (a,b) \cdot (a,c)$.
  Since $ax_1y_1 + bx_2y_1 + cx_1y_2$ and $x_2y_2$ are integers, we invoke \ref{1.48} to find $(a,bc) | \left( (a,b) \cdot (a,c) \right)$. Thus, $(a,bc) \leq (a,b) \cdot (a,c)$.

  We know that $(a,b) | a$ and that $(a, c) | a$. We also know that $\big( (a, b) , (a, c) \big) = 1$ (if it didn't, it would be a common factor of $b$ and $c$ greater than $1$ which is impossible). Thus, we cite \ref{1.42} to find $\left( (a,b) \cdot (a, c) \right) | a$. Thus, since it is a common factor, it is less than the greatest common factor, so we conclude $(a,b) \cdot (a,c) \leq (a, bc)$.

  Thus we have $(a,bc) \leq (a,b) \cdot (a,c) \leq (a, bc)$, so we conclude $(a,bc) = (a,b) \cdot (a,c)$.
\end{proof}



\begin{thm} \label{2.29}
  Let $a$, $b$, and $c$ be integers. If $(a,b) = 1$ and $(a,c) = 1$, then $(a, bc) = 1$
\end{thm}

\begin{proof}
  This is just \ref{1.43}.
\end{proof}



\begin{thm} \label{2.30}
  Let $a$ and $b$ be integers. If $(a,b) = d$, then $(\frac a d , \frac b d) = 1$.
\end{thm}

\begin{proof}
  
\end{proof}

\end{document}
