\documentclass{article}
\usepackage[utf8]{inputenc}
\usepackage[leqno]{mathtools}
\usepackage{amsthm}
\usepackage{amsfonts}
\usepackage[margin=0.75in]{geometry}

\DeclareMathOperator{\lcm}{lcm}

\newtheorem{thm}{Theorem}[section]
\newtheorem{ques}[thm]{Question}
\newtheorem{ex}[thm]{Exercise}
\newtheorem{cor}[thm]{Corollary}
\newtheorem{lem}{Lemma}[thm]
\newtheorem{PC}{Paul's Conjecture}

\numberwithin{equation}{thm}

\providecommand{\gmod}[1]{\; (\bmod \; #1)}

\usepackage{subfiles}

\title{Number Theory Notebook}
\author{Paul Schulze}
\date{January 22, 2021}

\begin{document}

\maketitle



\section{Chapter 1}


\subsection*{Divisibility and congruence}

\subfile{Chapter_1/Divisibility_and_Congruence}


\subsection*{The Division Algorithm}

\subfile{Chapter_1/The_Division_Algorithm}


\subsection*{Greatest common divisors and linear Diophantine equations}

\subfile{Chapter_1/GCD_and_linear_Diophantine}



\pagebreak



\section{Chapter 2}


\subsection*{Fundamental Theorem of Arithmetic}

\subfile{Chapter_2/Fundamental_Theorem_of_Arithmetic}


\subsection*{Applications of the Fundamental Theorem of Arithmetic}

\subfile{Chapter_2/Applications_of_the_FTA}


\subsection*{The infinitude of primes}

\subfile{Chapter_2/The_infinitude_of_primes}


\subsection*{Primes of special form}

\subfile{Chapter_2/Primes_of_special_form}


\subsection*{The distribution of primes}

\subfile{Chapter_2/The_distribution_of_primes.tex}



\pagebreak



\section{Chapter 3}


\subsection*{Powers and polynomials modulo $n$}

\subfile{Chapter_3/Powers_and_polynomials_modulo_n}


\subsection*{Linear congruences}

\subfile{Chapter_3/Linear_congruences}


\subsection*{Systems of linear congruences: \\ the Chinese Remainder Theorem}

\subfile{Chapter_3/Systems_of_linear_congruences}



\pagebreak



\section{Chapter 4}


\subsection*{Orders of an integer modulo $n$}

\begin{ex} \label{4.1}
  For $i = 0, 1, 2, 3, 4, 5$ and $6$, find the number in the CCRS to which $2^i$ is congruent modulo $7$.
\end{ex}

$1, 2, 4, 1, 2, 4, 1$.



\begin{thm} \label{4.2}
  Let $a$ and $n$ be natural numbers with $(a, n) = 1$. Then $(a^j, n) = 1$ for any natural number $j$.
\end{thm}

\begin{proof}
  I could've sworn we've done this already. In any case, the proof is by induction, with the base case being $j=1$ given to us immediately. The hypothesis is $(a^{j-1},n) = 1$, and this in conjunciton with the fact that $(a, n) = 1$ adn \ref{2.29} is enough to show $(a^j, n) = 1$.
\end{proof}



\begin{thm} \label{4.3}
  Let $a$, $b$, and $n$ be integers with $n > 0$ and $(a, n) = 1$. If $a \equiv b \gmod n$, then $(b, n) = 1$.
\end{thm}

\begin{proof}
  Since $a \equiv b \gmod n$, we know $n | (b - a)$. If $(b, n) > 1$, then since $(b, n) | b$ but $(b, n) \not | \; a$ (as if it did, it would be a common factor of $a$ and $n$ greater than $1$), we know $(b, n) \not | \; (b-a)$. However, $(b, n) | n$ and $n | (b-a)$, implying $(b, n) | (b-a)$, a contradiciton. Thus, it cannot be that $(b, n) > 1$, implying $(b, n) = 1$.
\end{proof}



\begin{thm} \label{4.4}
  Let $a$ and $n$ be natural numbers. Then there exist natural numbers $i$ and $j$, with $i \neq j$, such that $a^i \equiv a^j \gmod n$.
\end{thm}

\begin{proof}
  Consider the series $a^1, a^2, a^3, \ldots$. When we take this series and reduce each term to its corresponding member in the CCRS, there are only finitely many values each term can take, but infinitely many terms. Thus, by pigeonhole, two terms must take the same value, and thus two terms must be congruent modulo $n$.
\end{proof}



\begin{thm} \label{4.5}
  Let $a$, $b$, $c$, and $n$ be integers with $n > 0$. If $ac \equiv bc \gmod n$ and $(c, n) = 1$, then $a \equiv b \gmod n$.
\end{thm}

\begin{proof}
  $ac \equiv bc \gmod n$ implies $n | (bc - ac)$, implying $n | (c \cdot (b-a))$. Since $(c, n) = 1$, by \ref{1.41} we conclude $n | (b-a)$, implying $a \equiv b \gmod n$.
\end{proof}



\begin{thm} \label{4.6}
  Let $a$ and $n$ be natural numbers with $(a, n) = 1$. Then there exists a natural number $k$ such that $a^k \equiv 1 \gmod n$.
\end{thm}

\begin{proof}
  By \ref{4.4}, we know there exist distinct natural numbers $i$ and $j$ such that $a^i \equiv a^j \gmod n$. Assume WLOG $j > i$. By definition, this implies $n | (a^j - a^i)$, implying $n | (a^i \cdot (a^{j-i} - 1))$.

  Since $(a, n) = 1$, we invoke \ref{4.2} and find $(a^i, n = 1)$. Then, we combine that with $n | (a^i \cdot (a^{j-i} - 1))$ and invoke \ref{1.41} to conclude $n | (a^{j-i} - 1)$, which definitionally implies $a^{j-i} \equiv 1 \gmod n$ for some natural number $j-i$.
\end{proof}


\end{document}
