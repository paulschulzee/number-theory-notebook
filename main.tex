\documentclass{article}
\usepackage[utf8]{inputenc}
\usepackage[leqno]{mathtools}
\usepackage{amsthm}
\usepackage{amsfonts}
\usepackage[margin=0.75in]{geometry}

\newtheorem{thm}{Theorem}[section]
\newtheorem{ques}[thm]{Question}
\newtheorem{ex}[thm]{Exercise}
\newtheorem{lem}{Lemma}[thm]

\numberwithin{equation}{thm}

\providecommand{\gmod}[1]{\; (\bmod \; #1)}

\usepackage{subfiles}

\title{Number Theory Notebook}
\author{Paul Schulze}
\date{January 22, 2021}

\begin{document}

\maketitle



\section{Chapter 1}


\subsection*{Divisibility and congruence}

\subfile{Chapter_1/Divisibility_and_Congruence}


\subsection*{The Division Algorithm}

\subfile{Chapter_1/The_Division_Algorithm}


\subsection*{Greatest common divisors and linear Diophantine equations}



\begin{ques} \label{1.29}
  Do every two integers have at least one common divisor?
\end{ques}

Yes. For any two integers $a$ and $b$, $1 \cdot a = a$ and $1 \cdot b = b$ so $1 | a$ and $1 | b$, making $1$ a common divsor of $a$ and $b$.



\begin{ques} \label{1.30}
  Can two integers have infinitely many common divisors?
\end{ques}

No, if the two integers are distinct. Any nonzero integer $n$ can only have finitely many divisors, as any integer $d$ such that $d < -|n|$ or $d > |n|$ cannot be a divisor (since $1d$ and $-1d$ have a greater absolute value than $n$, and $0d = 0 \neq n$). In other words, only the numbers $f$ such that $-n \leq f \leq n$ are ``eligibile'' to be divisors of $n$, so there can only be finitely many divisors of $n$.



\begin{ex} \label{1.32}
  Find the following greatest common divisors. Which pairs are relatively prime?
\end{ex}

\hspace*{5mm} \emph{1. $(36, 22)$} \\
\hspace*{15mm} $2$ \\

\hspace*{5mm} \emph{2. $(45, -15)$} \\
\hspace*{15mm} $15$ \\

\hspace*{5mm} \emph{3. $(-296, -88)$} \\
\hspace*{15mm} $8$ \\

\hspace*{5mm} \emph{4. $(0, 256)$} \\
\hspace*{15mm} $256$ \\

\hspace*{5mm} \emph{5. $(15, 28)$} \\
\hspace*{15mm} $1$ (relatively prime) \\

\hspace*{5mm} \emph{6. $(1, -2436)$} \\
\hspace*{15mm} $1$ (relatively prime) \\



\begin{thm} \label{1.32}
  Let $a$, $n$, $b$, $r$, and $k$ be integers. If $a = nb + r$ and $k|a$ and $k|b$, then $k|r$.
\end{thm}

\begin{proof}
  Let $a = d_ak$ and $b=d_bk$, where $d_a$ and $d_b$ are the integers guaranteed by the facts that $k|a$ and $k|b$.
  Then, we have $d_ak = nd_bk + r$. Isolating $r$, we get $r = d_ak - nd_bk = k(d_a - nd_b)$. Since $n$, $d_a$, and $d_b$ are all integers, we know $d_a - nd_b$ is an integer. Thus, we've found $r$ is equal to $k$ times some integer, so $k|r$.
\end{proof}



\pagebreak



\begin{thm} \label{1.33}
  Let $a$, $b$, $n_1$, and $r_1$ be integers with $a$ and $b$ not both $0$. If $a = n_1b + r_1$, then $(a,b) = (b,r_1)$.
\end{thm}

\begin{proof}
  
\end{proof}

\end{document}
