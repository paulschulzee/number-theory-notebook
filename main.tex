\documentclass{article}
\usepackage[utf8]{inputenc}
\usepackage[leqno]{mathtools}
\usepackage{amsthm}
\usepackage{amsfonts}
\usepackage[margin=0.75in]{geometry}

\DeclareMathOperator{\lcm}{lcm}
\DeclareMathOperator{\ord}{ord}

\newtheorem{thm}{Theorem}[section]
\newtheorem{ques}[thm]{Question}
\newtheorem{ex}[thm]{Exercise}
\newtheorem{cor}[thm]{Corollary}
\newtheorem{lem}{Lemma}[thm]
\newtheorem{PC}{Paul's Conjecture}

\numberwithin{equation}{thm}

\providecommand{\gmod}[1]{\; (\bmod \; #1)}

\usepackage{subfiles}

\title{Number Theory Notebook}
\author{Paul Schulze}
\date{January 22, 2021}

\begin{document}

\maketitle



\section{Chapter 1}


\subsection*{Divisibility and congruence}

\subfile{Chapter_1/Divisibility_and_Congruence}


\subsection*{The Division Algorithm}

\subfile{Chapter_1/The_Division_Algorithm}


\subsection*{Greatest common divisors and linear Diophantine equations}

\subfile{Chapter_1/GCD_and_linear_Diophantine}



\pagebreak



\section{Chapter 2}


\subsection*{Fundamental Theorem of Arithmetic}

\subfile{Chapter_2/Fundamental_Theorem_of_Arithmetic}


\subsection*{Applications of the Fundamental Theorem of Arithmetic}

\subfile{Chapter_2/Applications_of_the_FTA}


\subsection*{The infinitude of primes}

\subfile{Chapter_2/The_infinitude_of_primes}


\subsection*{Primes of special form}

\subfile{Chapter_2/Primes_of_special_form}


\subsection*{The distribution of primes}

\subfile{Chapter_2/The_distribution_of_primes.tex}



\pagebreak



\section{Chapter 3}


\subsection*{Powers and polynomials modulo $n$}

\subfile{Chapter_3/Powers_and_polynomials_modulo_n}


\subsection*{Linear congruences}

\subfile{Chapter_3/Linear_congruences}


\subsection*{Systems of linear congruences: \\ the Chinese Remainder Theorem}

\subfile{Chapter_3/Systems_of_linear_congruences}



\pagebreak



\section{Chapter 4}


\subsection*{Orders of an integer modulo $n$}

\begin{ex} \label{4.1}
  For $i = 0, 1, 2, 3, 4, 5$ and $6$, find the number in the CCRS to which $2^i$ is congruent modulo $7$.
\end{ex}

$1, 2, 4, 1, 2, 4, 1$.



\begin{thm} \label{4.2}
  Let $a$ and $n$ be natural numbers with $(a, n) = 1$. Then $(a^j, n) = 1$ for any natural number $j$.
\end{thm}

\begin{proof}
  I could've sworn we've done this already. In any case, the proof is by induction, with the base case being $j=1$ given to us immediately. The hypothesis is $(a^{j-1},n) = 1$, and this in conjunciton with the fact that $(a, n) = 1$ adn \ref{2.29} is enough to show $(a^j, n) = 1$.
\end{proof}



\begin{thm} \label{4.3}
  Let $a$, $b$, and $n$ be integers with $n > 0$ and $(a, n) = 1$. If $a \equiv b \gmod n$, then $(b, n) = 1$.
\end{thm}

\begin{proof}
  Since $a \equiv b \gmod n$, we know $n | (b - a)$. Since $(b, n) | n$, we conclude $(b, n) | (b-a)$. This implies $(b, n) | (a-b)$, and since $(b, n) | b$ we conclude $(b, n) | a$. Then we know $(b, n)$ is a common factor of $a$ and $n$, and so we conclude $(b, n) \leq 1$. Thus, $(b, n) = 1$.
\end{proof}



\begin{thm} \label{4.4}
  Let $a$ and $n$ be natural numbers. Then there exist natural numbers $i$ and $j$, with $i \neq j$, such that $a^i \equiv a^j \gmod n$.
\end{thm}

\begin{proof}
  Consider the series $a^1, a^2, a^3, \ldots$. When we take this series and reduce each term to its corresponding member in the CCRS, there are only finitely many values each term can take, but infinitely many terms. Thus, by pigeonhole, two terms must take the same value, and thus two terms must be congruent modulo $n$.
\end{proof}



\begin{thm} \label{4.5}
  Let $a$, $b$, $c$, and $n$ be integers with $n > 0$. If $ac \equiv bc \gmod n$ and $(c, n) = 1$, then $a \equiv b \gmod n$.
\end{thm}

\begin{proof}
  $ac \equiv bc \gmod n$ implies $n | (bc - ac)$, implying $n | (c \cdot (b-a))$. Since $(c, n) = 1$, by \ref{1.41} we conclude $n | (b-a)$, implying $a \equiv b \gmod n$.
\end{proof}



\begin{thm} \label{4.6}
  Let $a$ and $n$ be natural numbers with $(a, n) = 1$. Then there exists a natural number $k$ such that $a^k \equiv 1 \gmod n$.
\end{thm}

\begin{proof}
  By \ref{4.4}, we know there exist distinct natural numbers $i$ and $j$ such that $a^i \equiv a^j \gmod n$. Assume WLOG $j > i$. By definition, this implies $n | (a^j - a^i)$, implying $n | (a^i \cdot (a^{j-i} - 1))$.

  Since $(a, n) = 1$, we invoke \ref{4.2} and find $(a^i, n = 1)$. Then, we combine that with $n | (a^i \cdot (a^{j-i} - 1))$ and invoke \ref{1.41} to conclude $n | (a^{j-i} - 1)$, which definitionally implies $a^{j-i} \equiv 1 \gmod n$ for some natural number $j-i$.
\end{proof}



\begin{ques} \label{4.7}
  Choose some relatively prime natural numbers $a$ and $n$ and compute the order of $a$ modulo $n$. Frame a conjecture concerning how large the order of $a$ modulo $n$ can be, depending on $n$.
\end{ques}

From Math Seminar I know $\ord_n (a) | n$.



\begin{thm} \label{4.8}
  Let $a$ and $n$ be natural numbers with $(a, n) = 1$ and let $k = \ord_n (a)$. Then the numbers $a^1, a^2, \ldots, a^k$ are pariwise incongruent modulo $n$.
\end{thm}

\begin{proof}
  Say there exist $1 \leq i, j \leq k$ such that $a^i \equiv a^j \gmod n$. WLOG, assume $j > i$. Then, notice that $n | (a^j - a^i)$ implies $n | (a^i \cdot (a^{j-i} - 1))$, which since $(n, a^i) = 1$ (\ref{4.2}) implies $n | (a^{j-i} - 1)$, implying $a^{j-i} \equiv 1 \gmod n$ for some $j-1 < k$.

  This is a contradiction, since $k$ is the \emph{smallest} integer such that $a^k \equiv 1$. Thus, our assumption that such $i$ and $j$ exist is flawed, and the numbers $a^1, a^2, \ldots, a^k$ are pariwise incongruent.
\end{proof}



\begin{thm} \label{4.9}
  Let $a$ and $n$ be natural numbers with $(a, n) = 1$ and let $k = \ord_n (a)$. For any natural number $m$, $a^m$ is congruent modulo $n$ to one of the numbers $a^1, a^2, \ldots, a^k$.
\end{thm}

\begin{proof}
  By division algorithm, $m = kq + r$ for some integers $q, r$ with $r < k$. Then, we find $a^m \equiv a^{kq + r} \equiv a^{kq} \cdot a^r \equiv (a^k)^q \cdot a^r \equiv 1^q \cdot a^r \equiv a^r \gmod n$ for some $0 \leq r \leq k-1$. Our one edge case is $a^m \equiv a^0 \equiv 1$, but we can clearly see this implies $a^m \equiv a^k$, and we're done.
\end{proof}



\pagebreak



\begin{thm} \label{4.10}
  Let $a$ and $n$ be natural numbers with $(a, n) = 1$, let $k = \ord_n (a)$, and let $m$ be a natural number. Then $a^m \equiv 1 \gmod n$ if and only if $k | m$.
\end{thm}

\begin{proof}
  Seeing that $k | m$ implies $a^m \equiv 1 \gmod n$ is easy: $a^m \equiv a^{kx} \equiv (a^k)^x \equiv 1^x \equiv 1 \gmod n$, for some integer $x$.

  To show $a^m \equiv 1 \gmod n$ implies $k | m$, we invoke the division algorithm and find $q, r$ such that $m = kq + r$ for some $0 \leq r < k$. Notice $1 \equiv a^m \equiv a^{kq + r} \equiv a^{kq} \cdot a^r \equiv (a^k)^q \cdot a^r \equiv 1^q \cdot a^r \equiv a^r \gmod n$.
  Since $a^r \equiv 1 \gmod n$, we conclude $r$ isn't between $1$ and $k-1$ inclusive, as then $k$ wouldn't be the \emph{smallest} number such that $a^k \equiv 1 \gmod n$. Thus, since $0 \leq r < k$, the only remaining possibility is that $r = 0$. Thus, $m = kq + r$ becomes $m = kq + 0 = kq$, implying $k | m$.
\end{proof}



\begin{thm} \label{4.11}
  Let $a$ and $n$ be natural numbers with $(a, n) = 1$. Then $\ord_n (a) < n$ (unless $n = 1$).
\end{thm}

\begin{proof}
  Suppose $\ord_n (a) \geq n$. Then, by \ref{4.8}, we know $a^1, a^2, \ldots, a^n$ are pairwise incongruent modulo $n$. This implies that each of these numbers reduces to a different number in the CCRS modulo $n$ (if two numbers reduced to the same thing, they'd be congruent). Thus, since the CCRS has $n$ elements like the sequence, every member of the CCRS must be ``hit'' by the powers of $a$. This implies there exists an $i$ such that $a^i \equiv 0 \gmod n$, or in other words $n | a^i$.
  Since $(a, n) = 1$, by \ref{4.2}we know $(a^i, n) = 1$ Since $n$ is a common factor of $n$ and $a^i$, we conclude $n = 1$. This is the one exception to the theorem: in any other case, we now know $\ord_n (a) < n$.
\end{proof}



\begin{ex} \label{4.12}
  Compute $a^{p-1} \gmod p$ for various numbers $a$ and primes $p$, and make a conjecture.
\end{ex}

$a = 4$, $p = 5$, $4^{5-1} \equiv 1 \gmod 5$.

$a = 3$, $p = 5$, $3^{5-1} \equiv 1 \gmod 5$.

$a = 4$, $p = 7$, $4^{7-1} \equiv 1 \gmod 7$.

$a = 3$, $p = 7$, $3^{7-1} \equiv 1 \gmod 7$.

It seems like $a^{p-1} \equiv 1 \gmod p$.



\begin{thm} \label{4.13}
  Let $p$ be a prime and let $a$ be an integer not divisible by $p$; that is, $(a, p) = 1$. Then $A = \{a, 2a, 3a, \ldots, pa\}$ is a complete residue system modulo $p$.
\end{thm}

\begin{proof}
  By \ref{3.17}, we only have to show that no two members of $A$ are congruent modulo $p$. To do this, take any two members $ia$ and $ja$, and assume WLOG that $j > i$. Since $j - i < p$, we know $p \not | \; (j-i)$, and we also know $p \not | \; a$. Thus, by \ref{2.27}, we know $p \not | \; (j-i)a$, or in other words $p \not | \; (aj - ai)$.
  By definition, this implies $aj \not \equiv ai \gmod p$, so the members of $A$ are pairwise incongruent and this (along with the fact that $A$ has $p$ elements) implies $A$ is a compelte residue system.
\end{proof}



\begin{thm} \label{4.14}
  Let $p$ be a prime and let $a$ be an integer not divisible by $p$. Then

  $$a \cdot 2a \cdot 3a \cdot \cdots \cdot (p-1)a \equiv 1 \cdot 2 \cdot 3 \cdot \cdots \cdot (p-1) \gmod p$$
\end{thm}

\begin{proof}
  By \ref{4.13}, we know the set $\{a, 2a, \ldots, pa\}$ is a CRS. Because of this, when we take the members of this set and ``reduce'' them to their CCRS members, we should exactly cover the CCRS. Now we kick out $pa$, since we know $pa \equiv 0 \gmod p$ (since $p | pa$). Goodbye, $pa$. What we're left with is a set that has one term equivalent to $1$ modulo $p$, one term equivalent to $2$ modulo $p$, etc., all the way up to a term equivalent to $p-1$ modulo $p$.

  Now, we take the product $a \cdot \cdots \cdot (p-1)a$ and replace each term with its corresponding meber of the CCRS. This won't change what it's congruent to mod $p$, since we're replacing things that are equivalent mod $p$. What we'll be left with, then, is $1 \cdot 2 \cdot \cdots \cdot p-1$, albeit probably not in that order.
\end{proof}



\begin{thm} \label{4.15}
  If $p$ is a prime and $a$ is an integer relatively prime to $p$, then $a^{p-1} \equiv 1 \gmod p$.
\end{thm}



\begin{thm} \label{4.16}
  If $p$ is a prime and $a$ is any integer, then $a^p \equiv a \gmod p$.
\end{thm}



\begin{thm} \label{4.17}
  The two versions of Fermat's Little Theorem above are equivalent.
\end{thm}

\begin{proof}
  Showing \ref{4.16} implies \ref{4.15} is as simple as realizing $a^p \equiv a \gmod p$ implies $a \cdot a^{p-1} \equiv a \cdot 1 \gmod p$ and invoking \ref{4.5}.

  Showing \ref{4.15} implies \ref{4.16} is as simples as multiply both sides of $a^{p-1} \equiv 1 \gmod p$ by $a$... for $a$ relatively prime to $p$. If $a$ is not relatively prime to $p$, that implies $p | a$, so $a^p \equiv 0 \equiv a \gmod p$, an easy edge case.
\end{proof}



\begin{thm} \label{4.18}
  Let $p$ be a prime and $a$ be an integer. If $(a, p) = 1$, then $\ord_p (a)$ divides $p-1$, that is, $\ord_p(a) | p-1$.
\end{thm}

\begin{proof}
  Since $a^{p-1} \equiv 1 \gmod p$ by \ref{4.15}, we cite \ref{4.10} and are done.
\end{proof}



\begin{ex} \label{4.19}
  Compute each of the followin without the aid of a calculator or computer.
\end{ex}

1. $512^{372} \equiv (512^{12})^{31} \equiv 1^{31} \equiv 1 \gmod{13}$.

2. $3444^{3233} \equiv (10^{16})^{202} \cdot 10^1 \equiv 1^{202} \cdot 10 \equiv 10 \gmod{17}$.

3. $123^{456} \equiv (8^{22})^{20} \cdot 8^16 \equiv 1^{20} \cdot 18^8 \equiv 2^4 \equiv 4^2 \equiv 8 \gmod{23}$.

\end{document}
