\documentclass{article}
\usepackage[utf8]{inputenc}
\usepackage[leqno]{mathtools}
\usepackage{amsthm}
\usepackage{amsfonts}
\usepackage[margin=0.75in]{geometry}

\DeclareMathOperator{\lcm}{lcm}
\DeclareMathOperator{\ord}{ord}

\newtheorem{thm}{Theorem}[section]
\newtheorem{ques}[thm]{Question}
\newtheorem{ex}[thm]{Exercise}
\newtheorem{cor}[thm]{Corollary}
\newtheorem{lem}{Lemma}[thm]
\newtheorem{PC}{Paul's Conjecture}

\numberwithin{equation}{thm}

\providecommand{\gmod}[1]{\; (\bmod \; #1)}

\usepackage{subfiles}

\title{Number Theory Notebook}
\author{Paul Schulze}
\date{January 22, 2021}

\begin{document}

\maketitle



\section{Chapter 1}


\subsection*{Divisibility and congruence}

\subfile{Chapter_1/Divisibility_and_Congruence}


\subsection*{The Division Algorithm}

\subfile{Chapter_1/The_Division_Algorithm}


\subsection*{Greatest common divisors and linear Diophantine equations}

\subfile{Chapter_1/GCD_and_linear_Diophantine}



\pagebreak



\section{Chapter 2}


\subsection*{Fundamental Theorem of Arithmetic}

\subfile{Chapter_2/Fundamental_Theorem_of_Arithmetic}


\subsection*{Applications of the Fundamental Theorem of Arithmetic}

\subfile{Chapter_2/Applications_of_the_FTA}


\subsection*{The infinitude of primes}

\subfile{Chapter_2/The_infinitude_of_primes}


\subsection*{Primes of special form}

\subfile{Chapter_2/Primes_of_special_form}


\subsection*{The distribution of primes}

\subfile{Chapter_2/The_distribution_of_primes.tex}



\pagebreak



\section{Chapter 3}


\subsection*{Powers and polynomials modulo $n$}

\subfile{Chapter_3/Powers_and_polynomials_modulo_n}


\subsection*{Linear congruences}

\subfile{Chapter_3/Linear_congruences}


\subsection*{Systems of linear congruences: \\ the Chinese Remainder Theorem}

\subfile{Chapter_3/Systems_of_linear_congruences}



\pagebreak



\section{Chapter 4}


\subsection*{Orders of an integer modulo $n$}

\subfile{Chapter_4/Orders_of_an_integer_modulo_n}


\subsection*{Fermat's Little Theorem}

\subfile{Chapter_4/Fermats_Little_Theorem}


\subsection*{Euler's Theorem and Wilson's Theorem}

\setcounter{thm}{26}

\begin{ques} \label{4.27}
  The numbers $1$, $5$, $7$, and $11$ are all natural numbers that are relatively prime prime to $12$, so $\phi (12) = 4$.
\end{ques}

What is $\phi (7)$? $6$.

What is $\phi (15)$? $8$.

What is $\phi (21)$? $12$.

What is $\phi (35)$? $24$.



\begin{thm} \label{4.28}
  Let $a$, $b$, and $n$ be integers such that $(a, n) = 1$ and $(b, n) = 1$. Then $(ab, n) = 1$.
\end{thm}

\begin{proof}
  Isn't this done with \ref{2.12}? Gross, don't make me type out all those subscripts again.
\end{proof}



\begin{thm} \label{4.29}
  Let $a$, $b$, and $n$ be integers with $n > 0$. If $a \equiv b \gmod n$ and $(a, n) = 1$, then $(b, n) = 1$.
\end{thm}

\begin{proof}
  $(b, n) | n$ and also since $n | (a - b)$ we conclude $(b, n) | (a - b)$. Since $(b, n) | b$, we know $(b, n) | ((a - b) + b)$, or in other words $(b, n) | a$.

  Then, since $(b, n) | n$ and $(b, n) | a$, its maximum value is $1$. Thus, $(b, n) = 1$.
\end{proof}



\begin{thm} \label{4.30}
  Let $a$, $b$, $c$, and $n$ be integers with $n > 0$. If $ab \equiv ac \gmod n$ and $(a, n) = 1$, then $b \equiv c \gmod n$.
\end{thm}

\begin{proof}
  Since $ab \equiv ac \gmod n$ we know $n | (ab - ac)$, so $n | a(b-c)$. If $(a, n) = 1$, we know by \ref{1.41} that $n | (b-c)$, i.e. $b \equiv c \gmod n$.
\end{proof}



\begin{thm} \label{4.31}
  Let $n$ be a natural number and let $x_1, x_2, \ldots, x_{\phi (n)}$ be the distinct natural numbers less than or equal to $n$ that are relatively prime to $n$. Let $a$ be a non-zero integer relatively prime to $n$ and let $i$ and $j$ be different natural numbers less than or equal to $\phi (n)$. Then $ax_i \not \equiv ax_j \gmod n$.
\end{thm}

\begin{proof}
  Assume WLOG $x_i > x_j$.

  Since $0 < x_i - x-j < n$, we know $n \not | \; (x_i - x_j)$. Since $(a, n) = 1$, we then know $n \not | \; a(x_i - x_j)$, i.e. $n \not | \; (ax_i - ax_j)$, i.e. $ax_i \not \equiv ax_j \gmod n$.
\end{proof}



\begin{thm} \label{4.32}
  If $a$ and $n$ are integers with $n > 0$ and $(a, n) = 1$, then

  $$a^{\phi (n)} \equiv 1 \gmod n .$$
\end{thm}

\begin{proof}
  Let $\Phi = \{x_1, x_2, \ldots, x_{\phi (n)} \}$ be the distinct natural numbers $\leq n$ that are relatively prime to $n$.

  To complete this proof, we will first show $ax_1 \cdot \cdots \cdot ax_{\phi (n)} \equiv x_1 \cdot \cdots \cdot x_{\phi (n)} \gmod n$.
\end{proof}




\end{document}
