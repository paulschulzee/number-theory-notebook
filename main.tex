\documentclass{article}
\usepackage[utf8]{inputenc}
\usepackage[leqno]{mathtools}
\usepackage{amsthm}
\usepackage{amsfonts}
\usepackage[margin=0.75in]{geometry}

\newtheorem{thm}{Theorem}[section]
\newtheorem{ques}[thm]{Question}
\newtheorem{ex}[thm]{Exercise}
\newtheorem{lem}{Lemma}[thm]

\numberwithin{equation}{thm}

\providecommand{\gmod}[1]{\; (\bmod \; #1)}

\usepackage{subfiles}

\title{Number Theory Notebook}
\author{Paul Schulze}
\date{January 22, 2021}

\begin{document}

\maketitle



\section{Chapter 1}


\subsection*{Divisibility and congruence}

\subfile{Chapter_1/Divisibility_and_Congruence}


\subsection*{The Division Algorithm}

\subfile{Chapter_1/The_Division_Algorithm}


\subsection*{Greatest common divisors and linear Diophantine equations}



\begin{ques}
  Do every two integers have at least one common divisor?
\end{ques}

Yes. For any two integers $a$ and $b$, $1 \cdot a = a$ and $1 \cdot b = b$ so $1 | a$ and $1 | b$, making $1$ a common divsor of $a$ and $b$.



\begin{ques}
  Can two integers have infinitely many common divisors?
\end{ques}

No, if the two integers are distinct. Any nonzero integer $n$ can only have finitely many divisors, as any integer $d$ such that $d < -|n|$ or $d > |n|$ cannot be a divisor (since $1d$ and $-1d$ have a greater absolute value than $n$, and $0d = 0 \neq n$). In other words, only the numbers $f$ such that $-n \leq f \leq n$ are ``eligibile'' to be divisors of $n$, so there can only be finitely many divisors of $n$.



\end{document}
