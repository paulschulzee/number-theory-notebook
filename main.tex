\documentclass{article}
\usepackage[utf8]{inputenc}
\usepackage[leqno]{mathtools}
\usepackage{amsthm}
\usepackage{amsfonts}
\usepackage[margin=0.75in]{geometry}

\DeclareMathOperator{\lcm}{lcm}

\newtheorem{thm}{Theorem}[section]
\newtheorem{ques}[thm]{Question}
\newtheorem{ex}[thm]{Exercise}
\newtheorem{cor}[thm]{Corollary}
\newtheorem{lem}{Lemma}[thm]
\newtheorem{PC}{Paul's Conjecture}

\numberwithin{equation}{thm}

\providecommand{\gmod}[1]{\; (\bmod \; #1)}

\usepackage{subfiles}

\title{Number Theory Notebook}
\author{Paul Schulze}
\date{January 22, 2021}

\begin{document}

\maketitle



\section{Chapter 1}

\subsection*{Divisibility and congruence}

\subfile{Chapter_1/Divisibility_and_Congruence}


\subsection*{The Division Algorithm}

\subfile{Chapter_1/The_Division_Algorithm}


\subsection*{Greatest common divisors and linear Diophantine equations}

\subfile{Chapter_1/GCD_and_linear_Diophantine}



\pagebreak



\section{Chapter 2}

\subsection*{Fundamental Theorem of Arithmetic}

\subfile{Chapter_2/Fundamental_Theorem_of_Arithmetic}


\subsection*{Applications of the Fundamental Theorem of Arithmetic}

\subfile{Chapter_2/Applications_of_the_FTA}


\subsection*{The infinitude of primes}

\subfile{Chapter_2/The_infinitude_of_primes}


\subsection*{Primes of special form}

\subfile{Chapter_2/Primes_of_special_form}


\subsection*{The distribution of primes}

\begin{thm} \label{2.46}
  There exist arbitraily long strings of consecutive composite numbers. That is, for any natural number $n$ there is a string of more than $n$ consecutive composite numbers.
\end{thm}

\begin{proof}
  For any $n$, define $k$ as follows.

  $$k = \prod_{p \in \mathbb{P}}^{n+1} p^{\lceil \log_{p}(n+1) \rceil}$$

  We will show that the string $k+2 \ldots k+n+1$ is a string of $n$ consecutive composite numbers. In other words, for all $i$ with $2 \leq i \leq (n+1)$, there exists some $p \neq k+i$ such that $p | (k+i)$.

  In fact, we will show that such a $p$ is less than or equal to $n$ (this obviously implies $p \neq k+i$ given that $k+i > n$).

  Let us examine the prime factorization of $i$. Since it is less than or equal to $n+1$, we know that its prime factors are also less than or equal to $n+1$. In fact, for any term $p_i ^ {r_i}$ in the prime factorization, we know $p_i ^ {r_i} \leq i \leq n+1 = p_i ^ {\log_{p_i}(n+1)}$, implying that $r_i \leq \log_{p_i}(n+1) \leq \lceil \log_{p_i}(n+1) \rceil$.
  By \ref{2.12}, we then notice that $i | k$ (as $k$'s prime factorization is equal to its definition above).
  Thus, we can write $k+i$ as $i \cdot (\frac{k}{i} + 1)$, where $\frac{k}{i} + 1$ is an integer. Thus we conclude $i | (k + i)$, and since $i \neq 1$ as $2 \leq 1$ and $i \neq k+1$ because $k \neq 0$ (something something irrelevant edge case where $n$ is small) we conclude that $k+i$ is composite, as it has a factor that is neither $1$ nor itself.

  Thus, the sequence $k+2 \ldots k+n+1$ is a string of $n$ consecutive composite numbers for any $n$.
\end{proof}


\end{document}
