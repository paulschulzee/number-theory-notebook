\documentclass{article}
\usepackage[utf8]{inputenc}
\usepackage[leqno]{mathtools}
\usepackage{amsthm}
\usepackage{amsfonts}
\usepackage[margin=0.75in]{geometry}

\DeclareMathOperator{\lcm}{lcm}

\newtheorem{thm}{Theorem}[section]
\newtheorem{ques}[thm]{Question}
\newtheorem{ex}[thm]{Exercise}
\newtheorem{cor}[thm]{Corollary}
\newtheorem{lem}{Lemma}[thm]
\newtheorem{PC}{Paul's Conjecture}

\numberwithin{equation}{thm}

\providecommand{\gmod}[1]{\; (\bmod \; #1)}

\usepackage{subfiles}

\title{Number Theory Notebook}
\author{Paul Schulze}
\date{January 22, 2021}

\begin{document}

\maketitle



\section{Chapter 1}

\subsection*{Divisibility and congruence}

\subfile{Chapter_1/Divisibility_and_Congruence}


\subsection*{The Division Algorithm}

\subfile{Chapter_1/The_Division_Algorithm}


\subsection*{Greatest common divisors and linear Diophantine equations}

\subfile{Chapter_1/GCD_and_linear_Diophantine}



\pagebreak



\section{Chapter 2}

\subsection*{Fundamental Theorem of Arithmetic}

\subfile{Chapter_2/Fundamental_Theorem_of_Arithmetic}


\subsection*{Applications of the Fundamental Theorem of Arithmetic}

\begin{thm} \label{2.12}
  Let $a$ and $b$ be natural numbers greater than $1$ and let $p_1^{r_1} p_2^{r_2} \cdots p_m^{r_m}$ be the unique prime factorization of $a$ and let $q_1^{t_1} q_2^{t_2} \cdots q_s^{t_s}$ be the unique prime factorization of $b$. Then $a | b$ if and only if for all $i \leq m$ there exists a $j \leq s$ such that $p_i = q_j$ and $r_i \leq t_j$.
\end{thm}

\begin{proof}
  Woo boy. Let's start by showing $a | b$ implies... all of that.

  We know that $a | b$ means $ak = b$. This means that $p_1^{r_1} \cdots p_m^{r_m} k = q_1^{t_1} \cdots q_s^{t_2}$. Since $k$ is an integer (and a natural number, given both $a$ and $b$ are natural) we know that it has its own unique prime factorization. Thus, the prime factorization of $a$ times the prime factorization of $k$ must be equal to the prime factorization of $b$ (since $ak = b$ and prime factorizations are unique).

  When we multiply the prime factorization of $a$ by that $k$, we cannot remove any of the terms $p_1 \ldots p_m$, nor can we reduce any of the exponents $r_1 \ldots r_m$, since the prime factorization of $k$ will not contain the multiplicative inverse of any prime. Thus, in order for our product to be the prime factorization of $b$, all of the primes $p_1 \ldots p_m$ must also be included in the prime factorization of $b$, and all of the exponents $r_1 \ldots r_m$ must be less than or equal to the corresponding exponents in the prime factorization of $b$. \\[0ex]

  Now the other direction. If we know that for all $i \leq m$ there exists a $j \leq s$ such that $p_1 = q_j$ and $r_1 \leq t_j$, then we can rewrite $b$ as $(p_1^{r_1} \cdots p_m^{r_m}) \cdot (q_1^{t_1'} \cdots q_s^{t_s'})$, where $t_1' \ldots t_s'$ are the exponents on the relative prime modified to accomodate ``moving'' $p_1^{r_1}$ through $p_m^{r_m}$ to the front of the product (these exponents notably may be $0$).
  Since $q_1^{t_1'} \cdots q_s^{t_s'}$ is an integer (call it $k$) and $p_1^{r_1} \cdots p_m^{r_m}$, we've shown that $b = ak$, or in other words $a | b$.
\end{proof}



\begin{thm} \label{2.13}
  If $a$ and $b$ are natural numbers and $a^2 | b^2$, then $a | b$.
\end{thm}

\begin{proof}
  Let $a = p_1^{r_1} \cdots p_m^{r_m}$ and $b = q_1^{t_1} \cdots q_s^{t_s}$ be the unique prime factorizations of these numbers.

  Notice that $a^2 = p_1^{2r_1} \cdots p_m^{2r_m}$ and $b^2 = q_1^{2t_1} \cdots q_s^{2t_s}$, and that these are the prime factorizations of these numbers.

  By \ref{2.12}, we find $a^2 | b^2$ implies that for all $i \leq m$ there exists a $j \leq s$ such that $p_i = q_j$ and $2r_i \leq 2t_j$. This implies that $r_i \leq t_j$, so we conclude that for all $i \leq m$ there exists a $j \leq s$ such that $p_i = q_j$ and $r_i \leq t_j$. By \ref{2.12}, this means $a | b$.
\end{proof}



\begin{ex} \label{2.14}
  Find $(3^14 \cdot 7^22 \cdot 11^5 \cdot 17^3, 5^2 \cdot 11^4 \cdot 13^8 \cdot 17)$
\end{ex}

$11^4 \cdot 17$



\begin{ex} \label{2.15}
  Find $\lcm (3^14 \cdot 7^22 \cdot 11^5 \cdot 17^3, 5^2 \cdot 11^4 \cdot 13^8 \cdot 17)$
\end{ex}

$3^14 \cdot 5^2 \cdot 7^22 \cdot 11^5 \cdot 13^8 \cdot 17^3$



\begin{ex} \label{2.16}
  Make a conjecture that generalizes the ideas you used to solve the two previous exercises.
\end{ex}

\begin{PC} \label{PC 2.16}
  Let the set of primes be denoted $P$, where $p_1 = 2$, $p_2 = 3$, $p_3 = 5$, $p_4 = 7$, $p_5 = 11$, etc.

  Let $a = p_1^{r_1} p_2^{r_2} \cdots$ and $b = p_1^{t_1} p_2^{t_2} \cdots$ be the prime factorizations of natural numbers $a$ and $b$, where $r_i$ and $t_j$ can be $0$ to indicate the absence of a prime. Then

  $$\gcd (a, b) = p_1^{\min(r_1, t_1)} p_2^{\min(r_2, t_2)} \cdots = \prod_{i \in \mathbb{N}} p_i^{\min(r_i, t_i)} \mbox{ and}$$
  $$\lcm (a, b) = p_1^{\max(r_1, t_1)} p_2^{\max(r_2, t_2)} \cdots = \prod_{i \in \mathbb{N}} p_i^{\max(r_i, t_i)}.$$
\end{PC}



\end{document}
