\documentclass{article}
\usepackage[utf8]{inputenc}
\usepackage[leqno]{mathtools}
\usepackage{amsthm}
\usepackage{amsfonts}
\usepackage[margin=0.75in]{geometry}

\DeclareMathOperator{\lcm}{lcm}
\DeclareMathOperator{\ord}{ord}

\newtheorem{thm}{Theorem}[section]
\newtheorem{ques}[thm]{Question}
\newtheorem{ex}[thm]{Exercise}
\newtheorem{cor}[thm]{Corollary}
\newtheorem{lem}{Lemma}[thm]
\newtheorem{PC}{Paul's Conjecture}

\numberwithin{equation}{thm}

\providecommand{\gmod}[1]{\; (\bmod \; #1)}

\usepackage{subfiles}

\title{Number Theory Notebook}
\author{Paul Schulze}
\date{January 22, 2021}

\begin{document}

\maketitle



\section{Chapter 1}


\subsection*{Divisibility and congruence}

\subfile{Chapter_1/Divisibility_and_Congruence}


\subsection*{The Division Algorithm}

\subfile{Chapter_1/The_Division_Algorithm}


\subsection*{Greatest common divisors and linear Diophantine equations}

\subfile{Chapter_1/GCD_and_linear_Diophantine}



\pagebreak



\section{Chapter 2}


\subsection*{Fundamental Theorem of Arithmetic}

\subfile{Chapter_2/Fundamental_Theorem_of_Arithmetic}


\subsection*{Applications of the Fundamental Theorem of Arithmetic}

\subfile{Chapter_2/Applications_of_the_FTA}


\subsection*{The infinitude of primes}

\subfile{Chapter_2/The_infinitude_of_primes}


\subsection*{Primes of special form}

\subfile{Chapter_2/Primes_of_special_form}


\subsection*{The distribution of primes}

\subfile{Chapter_2/The_distribution_of_primes.tex}



\pagebreak



\section{Chapter 3}


\subsection*{Powers and polynomials modulo $n$}

\subfile{Chapter_3/Powers_and_polynomials_modulo_n}


\subsection*{Linear congruences}

\subfile{Chapter_3/Linear_congruences}


\subsection*{Systems of linear congruences: \\ the Chinese Remainder Theorem}

\subfile{Chapter_3/Systems_of_linear_congruences}



\pagebreak



\section{Chapter 4}


\subsection*{Orders of an integer modulo $n$}

\subfile{Chapter_4/Orders_of_an_integer_modulo_n}


\subsection*{Fermat's Little Theorem}

\subfile{Chapter_4/Fermats_Little_Theorem}


\subsection*{An alternative route to Fermat's Little Theorem}

\subfile{Chapter_4/An_alternative_route_to_Fermats_Little_Theorem}


\subsection*{Euler's Theorem and Wilson's Theorem}

\subfile{Chapter_4/Eulers_Theorem_and_Wilsons_Theorem}



\pagebreak



\setcounter{section}{5}

\section{Chapter 6}

\subsection*{Lagrange's Theorem}

\begin{thm} \label{6.1}
  Let $f(x) = a_n x^n + a_{n-1} x^{n-1} + \cdots + a_0$ be a polynomial of degree $n > 0$ with integer coefficients and assume $a_n \neq 0$. Then an integer $r$ is a root of $f(x)$ if and only if htere exists a polynomial $g(x)$ of degree $n-1$ with integer coefficients such that $f(x) = (x-r)g(x)$.
\end{thm}

\begin{proof}
  To see that the existence of $g(x)$ such that $f(x) = (x-r) g(x)$ implies $r$ is a root is easy: notice $f(r) = (r-r) g(r) = 0 \cdot g(r) = 0$.

  The other direction requires some polynomial long division, which I won't Tex out because that would be a.) hell for me and b.) uninformative for the reader. Instead I will assert without evidence that $f(x) / (x-r)$ simplifies out to $a_n x^{n-1} + (a_n r + a_{n-1})x^{n-2} + \cdots + (a_n r^{n-1} + a_{n-1} r^{n-2} + \cdots + a_0) + a_n r^n + a_{n-1}r^{n-1} + \cdots + a_0$, with the part from $a_n r^n$ on being the remainder. Notice, then, that since $f(r) = 0$, the remainder is $0$. Looking at the coefficients on the rest of the terms, we notice they're all made up of integers added and multiplied together, and thus must all be integers. Thus, $f(x) / (x-r)$ is a satisfactory $g(x)$ such that $f(x) = (x-r) g(x)$.
\end{proof}



\begin{thm} \label{6.2}
  Let $f(x) = a_n x^n + a_{n-1} x^{n-1} + \cdots + a_0$ be a polynomial of degree $n > 0$ with integer coefficients and $a_n \neq 0$. Let $p$ be a prime number and $r$ an integer. Then, if $f(r) \equiv 0 \gmod p$, there exists a polynomial $g(x)$ of degree $n-1$ such that

  $$(x-r)g(x) = a_nx^n + a_{n-1}x^{n-1} + \cdots + a_1x + b_0$$

  where $a_0 \equiv b_0 \gmod p$
\end{thm}

\begin{proof}
  We divide $f(x) / (x-r)$ as above, but instead of noticing the remainder is $0$ we notice the remainder is congruent to $0$ modulo $p$. If we let the quotient (without the remainder) be $g(x)$, then we notice $f(x) = (x-r) g(x) + R$, where $R$ is the remainder. Rearranging, we obtain $(x-r) g(x) = f(x) - R = a_n x^n + \cdots + a_1 x + a_0 - R$.

  From here, we combine $a_0 - R$ into one term $b_0$, noticing that since $p | R$ we know $a_0 \equiv b_0 \gmod p$, and obtain $(x-r) g(x) = a_n x^n + \cdots + a_1 x + b_0$, and we're done.
\end{proof}



\begin{thm} \label{6.3}
  If $p$ is a prime and $f(x) = a_n x^n + a_{n-1} x^{n-1} + \cdots + a_0$ is a polynomial with integer coefficients and $a_n \neq 0 \gmod p$, then $f(x) \equiv 0 \gmod p$ has at most $n$ non-congruent solutions modulo $p$.
\end{thm}

\begin{proof}
  We will induct on $n$. Our base case is $n = 0$, in which we are trying to find solutions to $a_0 \equiv 0 \gmod p$ where $a_0 \not \equiv 0 \gmod p$. Obviously there are $0$ solutions to this.

  Our induction hypothesis is that for $m < n$, any polynomial $g(x) = a_m x^m + a_{m-1} x^{m-1} + \cdots + a_0$ has at most $m$ non-congruent modulo $p$ solutions to the equation $g(x) \equiv 0 \gmod p$. Our induction step hopes to show this property holds for $n$.

  Say $f(x) = a_n x^n + a_{n-1} x^{n-1} + \cdots a_0$. If the equation $f(x) \equiv 0 \gmod p$ has no solutions, we're done, so we'll go ahead and assume it has one solution $x = r$. Then by \ref{6.2} we find $f(x) \equiv (x-r) g(x)$ for some $g(x)$ of order less than $n$.

  Notice that any solution $k$ to $f(x) \equiv 0 \gmod p$ finds $f(k) \equiv (k-r) \cdot g(k) \equiv 0 \gmod p$. In other words, $p | (k-r)g(k)$. By \ref{2.27}, either $p | (k-r)$ (i.e. $k \equiv r \gmod p$) or $p | g(k)$ (i.e. $g(k) \equiv 0 \gmod p$).
  Thus, \emph{any solution to $f(x) \equiv 0 \gmod p$ that is incongruent to $r$ modulo $p$ must also be a solution to $g(x) \equiv 0 \gmod p$}. By our induction hypothesis, we know there are strictly less than $n$ pairwise incongruent modulo $p$ solutions to $g(x) \equiv 0 \gmod p$, so adding in $r$ means we know there are \emph{no more than} $n$ pairwise incongruent modulo $p$ solutions to  $f(x) \equiv 0 \gmod p$.
\end{proof}



\begin{thm} \label{6.4}
  Suppose $p$ is a prime and $\ord_p (a) = d$. Then for each natural number $i$ with $(i, d) = 1$, $\ord_p(a^i) = d$.
\end{thm}

\begin{proof}
  To find $k = \ord_p (a^i)$, we're looking for the smallest number such that $(a^i)^k \equiv 0 \gmod p$, or in other words $a^{ik} \equiv 0 \gmod p$. By \ref{4.10}, this is only the case when $d | ik$. Since $(i, d) = 1$, we know $d | ik$ implies $d | k$ (easily shown by \ref{2.12}. I swear to God we proved this somewhere, but this is like the 5th time I've gone to cite this imaginary theorem and I never find it). The smallest (natural) $k$ such that $d | k$ is $d$, so $k = \ord_p (a^i) = d$.
\end{proof}



\pagebreak



\begin{thm} \label{6.5}
  For a prime $p$ and a natural number $d$, at most $\phi (d)$ incongruent integers modulo $p$ have order $d$ modulo $p$.
\end{thm}

\begin{proof}
  Either there is an integer with order $d$, or there isn't. If there isn't we're done, so let's assume there is one and call it $a$.

  we will show that all numbers with order $d$ are congruent to an integer of the form $a^x$ where $(d, x) = 1$ and $x < d$. As there are only $\phi(d)$ possible values of $x$, this will mean that all integers with order $d$ will be congruent to one of $\phi(x)$ integers, and thus there can at most be $\phi(x)$ incongruent integers with order $d$.

  Let $k$ be an integer with order $d$. Let $A = \{a, a^2, a^3, \ldots, a^d\}$ and $K = \{a, a^2, a^3, \ldots, a^d\}$. By \ref{4.8}, we know both of these sets are pairwise incongruent. We also know that every element of $A$ is a solution to the equation $x^d - 1 \equiv 0 \gmod p$, as $(a^i)^d - 1 \equiv (a^d)^i - 1 \equiv 1^i - 1 \equiv 0 \gmod p$. Similarly for $K$.
  By \ref{6.3}, there can only be $d$ incongruent solutions to this equation. Thus, since $A$ is full of $d$ pairwise incongruent integers that are all solutions, it must contain one number congruent to each solution. Same with $K$. Thus, for any element of $A$, there must be a congruent element in $K$, and vice versa. Since $k \in K$, we know $k = a^i$ for some $i < d$.

  To see that $(d, i) = 1$, we notice that $k^{d/(d,i)} \equiv a^{d \cdot \frac{i}{(d,i)}} \equiv 1 \gmod p$ (\ref{4.10}) Since the order of $k$ is $d$, we then know $d/(d,i) \geq d$, implying $1 \geq (d,i)$, which means $(d,i) = 1$.
\end{proof}



\begin{thm} \label{6.6}
  Let $p$ be a prime and suppose $g$ is a primitive root modulo $p$. Then the set $\{0, g, g^2, g^3, \ldots, g^{p-1}\}$ forms a CRS modulo $p$.
\end{thm}

\begin{proof}
  Since this set has $p$ elements, by \ref{3.17}, all we have to do to show this is a CRS is to show that the terms are pairwise incongruent modulo $p$.

  None of the powers will ever be congruent to $0$, because that would imply $p | g^n$ which is impossible since no $p$ can appear in the prime factorization of $g^n$ (as if it did, $g \equiv 0 \gmod p$ and then $g$ has no order).

  Say there are two integers $i, j$ such that $1 \leq i, j \leq p-1$ and $g^i \equiv g^j \gmod p$. Assume WLOG $i > j$. Then $g^i \equiv g^j \gmod p$ implies $g^{i-j} \cdot g^j \equiv 1 \cdot g^j \gmod p$, which by \ref{4.5} means $g^{i-j} \equiv 1 \gmod p$.
  By \ref{5.10}, this means $(p-1) | (i - j)$, but since $0 \leq i - j \leq p-2$ the only multiple of $p-1$ that the difference could be is $0$, implying $i = j$. Thus, if two of the nonzero members of the set are congruent, they are the same element. In other words, they are pairwise incongruent modulo $p$.
\end{proof}



\begin{ex} \label{6.7}
  For each of the primes $p$ less than $20$ find a primitive root and make a chart showing what powers of the primitive root give each of the natural numbers less than $p$.
\end{ex}

$2$: $1^1 \equiv 1$.

$3$: $2^2 \equiv 1$, $2^1 \equiv 2$.

$5$: $3^4 \equiv 1$, $3^3 \equiv 2$, $3^1 \equiv 3$, $3^2 \equiv 4$.

$7$: $3^6 \equiv 1$, $3^2 \equiv 2$, $3^1 \equiv 3$, $3^4 \equiv 4$, $3^5 \equiv 5$, $3^3 \equiv 6$.

$11$: $2^{10} \equiv 1$, $2^1 \equiv 2$, $2^8 \equiv 3$, $2^2 \equiv 4$, $2^4 \equiv 5$, $2^9 \equiv 6$, $2^7 \equiv 7$, $2^3 \equiv 8$,
$2^6 \equiv 9$, $2^5 \equiv 10$.

$13$: $2^{12} \equiv 1$, $2^1 \equiv 2$, $2^4 \equiv 3$, $2^2 \equiv 4$, $2^9 \equiv 5$, $2^5 \equiv 6$, $2^{11} \equiv 7$, $2^3 \equiv 8$,
$2^8 \equiv 9$, $2^{10} \equiv 10$, $2^7 \equiv 11$, $2^6 \equiv 12$.

$17$: $3^{16} \equiv 1$, $3^{11} \equiv 2$, $3^1 \equiv 3$, $3^{12} \equiv 4$, $3^5 \equiv 5$, $3^{15} \equiv 6$, $3^{11} \equiv 7$, $3^{10} \equiv 8$,
$3^2 \equiv 9$, $3^3 \equiv 10$, $3^7 \equiv 11$, $3^{13} \equiv 12$, $3^4 \equiv 13$, $3^9 \equiv 14$, $3^6 \equiv 15$, $3^8 \equiv 16$.

$19$: $2^{18} \equiv 1$, $2^1 \equiv 2$, $2^{13} \equiv 3$, $2^2 \equiv 4$, $2^{16} \equiv 5$, $2^{14} \equiv 6$, $2^6 \equiv 7$, $2^3 \equiv 8$,
$2^8 \equiv 9$, $2^{17} \equiv 10$, $2^{12} \equiv 11$, $2^{15} \equiv 12$, $2^5 \equiv 13$, $2^7 \equiv 14$, $2^{11} \equiv 15$, $2^4 \equiv 16$,
$2^{10} \equiv 17$, $2^9 \equiv 18$.



\begin{thm} \label{6.8}
  Every prime $p$ has a primitive root.
\end{thm}

\begin{proof}
  It says we'll come back to this one.
\end{proof}



\begin{ex} \label{6.9}
  Consider the prime $p = 13$. For each divisor $d = 1, 2, 3, 4, 6, 12$ of $12 = p-1$, mark which of the natural numbers in the set $\{1, 2, 3, \ldots, 12\}$ have order $d$.
\end{ex}

Order $1$: Just $1$.

Order $2$: Just $12$.

Order $3$: $3$, $9$.

Order $4$: $5$, $8$.

Order $6$: $4$, $10$.

Order $12$: $2$, $6$, $7$, $11$.



\begin{ex} \label{6.10}
  Compute each of the following sums
\end{ex}

1. $\sum_{d | 6} \phi(d) = 1 + 1 + 2 + 2 = 6$

2. $\sum_{d|10} \phi(d) = 1 + 1 + 4 + 4 = 10$

3. $\sum_{d|24} \phi(d) = 1 + 1 + 2 + 2 + 2 + 4 + 4 + 8 = 24$

4. $\sum_{d|36} \phi(d) = 1 + 1 + 2 + 2 + 2 + 6 + 4 + 6 + 12 = 36$

5. $\sum_{d|27} \phi(d) = 1 + 2 + 6 + 18 = 27$

\emph{Make a sweeping conjecture about the sum of $\phi(d)$ taken over all natural divisors of any natural number n.}

\begin{PC} \label{PC 6.18}
  For any natural number $n$,

  $$\sum_{d | n} \phi(d) = n$$
\end{PC}



\begin{thm} \label{6.11}
  If $p$ is a prime, then

  $$\sum_{d|p} \phi(d) = p$$
\end{thm}

\begin{proof}
  $\sum_{d|p} \phi(d) = \phi(1) + \phi(p) = 1 + (p-1) = p$.
\end{proof}



\begin{thm} \label{6.12}
  If $p$ is a prime, then

  $$\sum_{d|p^k} \phi(d) = p^k$$
\end{thm}

\begin{proof}
  First, notice $\phi(p^i) = p^i - p^{i-1}$ for any natural $i$. This is because there are $p^i$ natural numbers less than or equal to $p^i$, and 1 in every $p$ of them is a multiple of $p$ and thus not relatively prime (i.e. $p^{i} / p = p^{i-1}$ of them are multiple of $p$).

  $\sum_{d|p^k} \phi(d) = \sum_{i=0}^k \phi(p^i) = 1 + \sum_{i=1}^k p^i - p^{i-1} = p^k$.
\end{proof}



\begin{thm} \label{6.13}
  If $p$ and $q$ are two different primes, then

  $$\sum_{d|pq} \phi(d) = pq$$
\end{thm}

\begin{proof}
  There are $pq$ natural numbers less than or equal to $pq$. One in $p$ are a multiple of $p$, one in $q$ are a multiple of $q$, and the rest are relatively prime to $pq$. Thus $\phi(pq) = pq - q - p + 1$ (the $+1$ is because we subtracted out $pq$ twice, once as a multiple of $p$ and once as a multiple of $q$.

  $\sum_{d|pq} \phi(d) = \phi(1) + \phi(p) + \phi(q) + \phi(pq) = 1 + (p-1) + (q-1) + (pq - q - p + 1) = pq$
\end{proof}



\begin{thm} \label{6.14}
  If $n$ and $m$ are relatively prime natural numbers, then

  $$\left( \sum_{d|m} \phi(d) \right) \cdot \left( \sum_{d|n} \phi(d) \right) = \sum_{d|mn} \phi(d)$$
\end{thm}

\begin{proof}
  6.23 kills this problem, and I have a proof of it, but I'll come back here and do it once I have that.
\end{proof}



\begin{thm} \label{6.15}
  If $n$ is a natural number, then

  $$\sum_{d|n} \phi(d) = n$$
\end{thm}

\begin{proof}
  By FTA, any $n = p_1^{r_1} \cdot p_2^{r_2} \cdot \cdots \cdot p_m^{r_m}$ for distinct primes $p_1$ through $p_m$ and natural numbers $r_1$ through $r_m$. We will induct on $m$.

  Our base case is $m = 0$, which is \ref{6.12}.

  Our inductive hypothesis is that for all $k < m$, if $n = p_1^{r_1} \cdot \cdots \cdot p_k^{r_k}$ then $\sum_{d|n} \phi(d) = n$. Our inductive step must show that this holds for $m$.

  Say $n = p_1^{r_1} \cdot \cdots \cdot p_{m-1}^{r_{m-1}} \cdot p_m^{r_m}$. Let $\eta = p_1^{r_1} \cdot \cdots \cdot p_{m-1}^{r_{m-1}}$. Notice by our inductive hypothesis that $\sum_{d|\eta} \phi(d) = \eta$ and $\sum_{d|p_m^{r_m}} \phi(d) = p_m^{r_m}$.
  Also notice by \ref{2.12} that $(\eta, p_m^{r_m}) = 1$. Then, by \ref{6.14} we see $n = \eta \cdot p_m^{r_m} = \left( \sum_{d|\eta} \phi(d) \right) \cdot \left( \sum_{d|p_m^{r_m}} \phi(d) \right) = \sum_{d|\eta \cdot p_m^{r_m}} \phi(d) = \sum_{d|n} \phi(d)$.

  Removing the middle steps, we find $n = \sum_{d|n} \phi(d)$.
\end{proof}



\begin{ex} \label{6.16}
  For a natrual number $n$ consider the fractions

  $$\frac{1}{n}, \frac{2}{n}, \frac{3}{n}, \ldots, \frac{n}{n},$$

  all written in reduced form. Try to find a natural one-tone correspondence between the reduced fractions and the number $\phi(d)$ for $d | n$. Show how that ovservation provides a very clever proof to the preceding theorem.
\end{ex}

\begin{proof}
  For any $d | n$, there are $\phi(d)$ fractions that have $d$ as their denominator in simplest form.

  Notice that for any $d | n$, the list of fractions above will contain all of $\frac{1}{d}, \frac{2}{d}, \ldots, \frac{d}{d}$. This is because for any $u \leq d$, we know $\frac{u}{d} \cdot (n/d) = \frac{u(n/d)}{n}$, and $u(n/d) = n(u/d) \leq n$ because $u/d \leq 1$.

  Thus, there are $d$ ``candidates'' for fractions with $d$ as their denominator in simplest form. Notice that since simplest form is when the numerator and denominator are relatively prime, these fractions do have $d$ as their denominator in simplest form exactly when the numerator is relatively prime to $d$. Since the numerator can be any value from $1$ to $d$, this means the number of fractions with $d$ as the denominator in simplest form is equal to the number of natruals form $1$ to $d$ that are relatively prime to $d$, or $\phi(d)$.

  Notice, then, that every fraction in the series $\frac{1}{n}, \ldots, \frac{n}{n}$ has to have a simplest form (with some divisor of $n$ as the denominator). Thus, if we add up the number of fractions with $d$ as their denominator in simplest form for all $d | n$, we'll get the number of fractions (which is $n$, since there are $n$ possibilities for the numerator). In other words, since $\phi(d)$ is the number of fractions with $d$ in their denominator in simplest form, we're saying $n = \sum_{d | n} \phi(d)$.
\end{proof}



\begin{thm} \label{6.17}
  Every prime $p$ has $\phi(p-1)$ primitive roots.
\end{thm}



\end{document}
