\documentclass{article}
\usepackage[utf8]{inputenc}
\usepackage[leqno]{mathtools}
\usepackage{amsthm}
\usepackage{amsfonts}
\usepackage[margin=0.75in]{geometry}

\DeclareMathOperator{\lcm}{lcm}

\newtheorem{thm}{Theorem}[section]
\newtheorem{ques}[thm]{Question}
\newtheorem{ex}[thm]{Exercise}
\newtheorem{cor}[thm]{Corollary}
\newtheorem{lem}{Lemma}[thm]
\newtheorem{PC}{Paul's Conjecture}

\numberwithin{equation}{thm}

\providecommand{\gmod}[1]{\; (\bmod \; #1)}

\usepackage{subfiles}

\title{Number Theory Notebook}
\author{Paul Schulze}
\date{January 22, 2021}

\begin{document}

\maketitle



\section{Chapter 1}


\subsection*{Divisibility and congruence}

\subfile{Chapter_1/Divisibility_and_Congruence}


\subsection*{The Division Algorithm}

\subfile{Chapter_1/The_Division_Algorithm}


\subsection*{Greatest common divisors and linear Diophantine equations}

\subfile{Chapter_1/GCD_and_linear_Diophantine}



\pagebreak



\section{Chapter 2}


\subsection*{Fundamental Theorem of Arithmetic}

\subfile{Chapter_2/Fundamental_Theorem_of_Arithmetic}


\subsection*{Applications of the Fundamental Theorem of Arithmetic}

\subfile{Chapter_2/Applications_of_the_FTA}


\subsection*{The infinitude of primes}

\subfile{Chapter_2/The_infinitude_of_primes}


\subsection*{Primes of special form}

\subfile{Chapter_2/Primes_of_special_form}


\subsection*{The distribution of primes}

\subfile{Chapter_2/The_distribution_of_primes.tex}



\pagebreak



\section{Chapter 3}


\subsection*{Powers and polynomials modulo $n$}

\subfile{Chapter_3/Powers_and_polynomials_modulo_n}


\subsection*{Linear congruences}

\subfile{Chapter_3/Linear_congruences}


\subsection*{Systems of linear congruences: \\ the Chinese Remainder Theorem}


\begin{ex} \label{3.25}
  A band of $17$ pirates stole a sack of gold coins. When they tried to divide the fortune into equal portions, $3$ coins remained. In the ensuing brawl over who should get the extra coins, one pirate was killed. The coins were redistributed, but this time an equal division left $10$ coins. Again they fought about who shuld get the remaining coins and another pirate was killed. Now, fortunately, the coins could be divided evenly among the surviving $15$ pirates. What was the fewest number of coinst taht could have been in the sack?
\end{ex}

This would never happen. In all likelihood, the coins would simply be put in the ship's treasury to pay for injured pirates' medical expenses and/or send money to family of pirates killed in battle.

The number of coins must be a multiple of $15$. Call it $15k$.

Then, we know $15 \equiv 10 \gmod{16}$. From this, since $15 \equiv -1$ and $10 \equiv -6$, we find $(-1) k \equiv -6$. Multiplying both sides by $-1$, we get $k \equiv 6 \gmod{16}$.

Similarly, $15k \equiv 3 \gmod{17}$, so $(-2)k \equiv -14$. Multiplying both sides by $8$, we get $k \equiv -112 \equiv 7 \gmod{17}$.

Since $k \equiv 6 \gmod{16}$, we know $k$ is of the form $16x + 6$. Since $k \equiv 7 \gmod{17}$, we conclude $16x \equiv 7 \gmod{17}$, implying $16x \equiv 1$. Then, since $16 \equiv -1 \gmod{17}$, we conclud $-1x \equiv 1 \gmod{17}$, or $x \equiv -1 \equiv 16 \gmod{17}$.

Since we're trying to minimize $15k$, we're trying to minimize $k$. Similarly, that means we're trying to minimize $16x + 6$, so we can obtain our answer by minimizing $x$. The smallest positive integer (since there are more than $0$ coins) that is equivalent to $16$ modulo $17$ is $16$, so plugging that in we obtain $k = 16 \cdot 16 + 6 = 262$ and then $15k = 15 \cdot 262 = 3930$.



\begin{ex} \label{3.26}
  WHen eggs in a basket are removed two, three, four, five, or six at a time, there remain, respectively, one, two, three, four, or five eggs. When they are taken out seven at a time, none are left over. Find the smallest number of eggs that could have been contained in the basket.
\end{ex}

Notice that if we added one more egg, the number of eggs would be divisble by two, three, four, five, and six. Thus, its prime factorization would have to include $2^2 3^1 5$. In other words, it would have to be multiple of $60$. Thus, the number of eggs has to be one less than a multiple of $60$. Call it $60x - 1$.

We know $7 | (60x-1)$. Thus, $60x \equiv 1 \gmod 7$. Noticing $60 \equiv 4 \gmod 7$, we conclude $4x \equiv 1 \gmod 7$. Multiplying both sides by $2$ we obtain $x \equiv 2 \gmod 7$. To minimze $60x - 1$, we minmiize $x$: in this case, the smallest possible value is $x = 2$. Plugging that in, we get $60 \cdot 2 - 1 = 119$.



\begin{thm} \label{3.27}
  Let $a$, $b$, $m$, and $n$ be integers with $m > 0$ and $n > 0$. Then the system
  \begin{gather*}
    x \equiv a \gmod n \\
    x \equiv b \gmod m
  \end{gather*}
  has a solution if and only if $(n, m) | a-b$.
\end{thm}

\begin{proof}
  Every step in this proof will be bidirectional.

  $x \equiv a \gmod n$ and $x \equiv b \gmod m$ can be rewritten as $ny = x - a$ and $mz = x - b$ for some integers $y$ and $z$. This implies $mz = ny + a - b$, rewritten as $n(-y) + mz = a - b$ for some integers $y$ and $z$. By \ref{1.48}, said integers $y$ and $z$ exist if and only if $(n, m)| a - b$.
\end{proof}



\begin{thm} \label{3.28}
  Let $a$, $b$, $m$, and $n$ be integers with $m . 0$, $n > 0$, and $(m, n) = 1$. Then the system
  \begin{gather*}
    x \equiv a \gmod n \\
    x \equiv b \gmod m
  \end{gather*}
  has a unique solution modulo $mn$.
\end{thm}

\begin{proof}
  We will conjure two solutions to this system, $x_1$ and $x_2$, and show they are congruent modulo $nm$.

  Notice $x_1 \equiv a \gmod n$ implies $ny_1 = x_1 - a$ for some integer $y_1$. Similarly, we find $mz_1 = x_1 - b$, $ny_2 = x_2 - a$, and $nz_2 = x_2 - b$.

  Take $ny_1 = x_1 - a$ and subtract $ny_2 = x_2 - a$ to obtain $ny_1 - ny_2 = x_1 - a - (x_2 - a)$. With some rearrangement, we find $n(y_1 - y_2) = x_1 - x_2$, so we conclude $n | (x_1 - x_2)$.

  Similarly, taking $mz_1 = x_1 - b$ and subtracting $mz_2 = x_2 - b$ gives us $m(z_1 - z_2) = x_1 - x_2$, implying $m | (x_1 - x_2)$.

  By \ref{1.42}, since $(n, m) = 1$, we conclude $nm | (x_1 - x_2)$. By definition, this means $x_1 \equiv x_2 \gmod{nm}$, and we're done.
\end{proof}



\begin{thm} \label{3.29}
  Suppose $n_1, n_2, \ldots, n_L$ are positive integers that are pairwise relatively prime, that is, $(n_1, n_j) = 1$ for $i \neq j$, $1 \leq i, j \leq L$. Then the system of $L$ congruences
  \begin{gather*}
    x \equiv a_1 \gmod{n_1} \\
    x \equiv a_2 \gmod{n_2} \\
    \vdots \\
    x \equiv a_L \gmod{n_L}
  \end{gather*}
  has a unique solution modulo the product $n_1n_2n_3\cdots n_L$.
\end{thm}



\end{document}
