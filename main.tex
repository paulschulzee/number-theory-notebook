\documentclass{article}
\usepackage[utf8]{inputenc}
\usepackage[leqno]{mathtools}
\usepackage{amsthm}
\usepackage{amsfonts}
\usepackage[margin=0.75in]{geometry}

\newtheorem{thm}{Theorem}[section]
\newtheorem{ques}[thm]{Question}
\newtheorem{ex}[thm]{Exercise}
\newtheorem{lem}{Lemma}[thm]
\newtheorem{PC}{Paul's Conjecture}

\numberwithin{equation}{thm}

\providecommand{\gmod}[1]{\; (\bmod \; #1)}

\usepackage{subfiles}

\title{Number Theory Notebook}
\author{Paul Schulze}
\date{January 22, 2021}

\begin{document}

\maketitle



\section{Chapter 1}


\subsection*{Divisibility and congruence}

\subfile{Chapter_1/Divisibility_and_Congruence}


\subsection*{The Division Algorithm}

\subfile{Chapter_1/The_Division_Algorithm}


\subsection*{Greatest common divisors and linear Diophantine equations}



\begin{ques} \label{1.29}
  Do every two integers have at least one common divisor?
\end{ques}

Yes. For any two integers $a$ and $b$, $1 \cdot a = a$ and $1 \cdot b = b$ so $1 | a$ and $1 | b$, making $1$ a common divsor of $a$ and $b$.



\begin{ques} \label{1.30}
  Can two integers have infinitely many common divisors?
\end{ques}

No, if the two integers are distinct. Any nonzero integer $n$ can only have finitely many divisors, as any integer $d$ such that $d < -|n|$ or $d > |n|$ cannot be a divisor (since $1d$ and $-1d$ have a greater absolute value than $n$, and $0d = 0 \neq n$). In other words, only the numbers $f$ such that $-n \leq f \leq n$ are ``eligibile'' to be divisors of $n$, so there can only be finitely many divisors of $n$.



\begin{ex} \label{1.31}
  Find the following greatest common divisors. Which pairs are relatively prime?
\end{ex}

\hspace*{5mm} \emph{1. $(36, 22)$} \\
\hspace*{15mm} $2$ \\

\hspace*{5mm} \emph{2. $(45, -15)$} \\
\hspace*{15mm} $15$ \\

\hspace*{5mm} \emph{3. $(-296, -88)$} \\
\hspace*{15mm} $8$ \\

\hspace*{5mm} \emph{4. $(0, 256)$} \\
\hspace*{15mm} $256$ \\

\hspace*{5mm} \emph{5. $(15, 28)$} \\
\hspace*{15mm} $1$ (relatively prime) \\

\hspace*{5mm} \emph{6. $(1, -2436)$} \\
\hspace*{15mm} $1$ (relatively prime) \\



\begin{thm} \label{1.32}
  Let $a$, $n$, $b$, $r$, and $k$ be integers. If $a = nb + r$ and $k|a$ and $k|b$, then $k|r$.
\end{thm}

\begin{proof}
  Let $a = d_ak$ and $b=d_bk$, where $d_a$ and $d_b$ are the integers guaranteed by the facts that $k|a$ and $k|b$.
  Then, we have $d_ak = nd_bk + r$. Isolating $r$, we get $r = d_ak - nd_bk = k(d_a - nd_b)$. Since $n$, $d_a$, and $d_b$ are all integers, we know $d_a - nd_b$ is an integer. Thus, we've found $r$ is equal to $k$ times some integer, so $k|r$.
\end{proof}



\pagebreak



\begin{thm} \label{1.33}
  Let $a$, $b$, $n_1$, and $r_1$ be integers with $a$ and $b$ not both $0$. If $a = n_1b + r_1$, then $(a,b) = (b,r_1)$.
\end{thm}

\begin{proof}
  We will show that the common divisors of $a$ and $b$ are the same as the common divisors of $b$ and $r_1$, and thus conclude that the greatest element of $S$ is also the greatest element of $T$.

  Let $S$ be the set of common divisors of $a$ and $b$, and let $T$ be the set of common divisors of $b$ and $r_1$. We will show $S = T$ by double inclusion.

  First, let's show $S \subset T$. Take an arbitrary $s \in S$. Since $s|a$ and $s|b$, we conclude $\exists d_a, d_b \in \mathbb{Z} \ni a = sd_a, b=sd_b$. We can then rearrange $a = n_1b + r_1$ to read $r_1 = a - n_1b$, and then plug in our previous two equations to get $r_1 = sd_a - n_1 sd_b \implies r_1 = s(d_a - n_1 d_b)$.
  Since $d_a$, $d_b$, and $n_1$ are all integers, we know $d_a - n_1 d_b$ is an integer, thus implying that $s | r_1$. Since we know $s|b$ since $s \in S$, we conclude $s \in T$. Thus, any arbitrary $s \in S$ is an element of $T$, so $S \subset T$.

  Showing that $T \subset S$ proceeds in much the same way. Take $t \in T$, conclude since $t|b$ and $t|r_1$ we find $\exists d_b d_r \in \mathbb{Z} \ni b = td_b, r_1 = td_r$, and then plug those in to $a = n_1b + r_1$ to get $a = n_1 td_b + td_r \implies a = t(n_1 d_b + d_r)$. Since $n_1$, $d_b$, and $d_r$ are integers, we find $t | a$, and since $t | b$ because $t \in T$, we thus conclude $t \in S$.
  Thus any arbitrary $t \in T$ is an element of $S$, so $T \subset S$.

  Thus, by double inclusion, $S = T$. This implies that the greates element of $S$, i.e. $(a, b)$, is equal to the greatest element of $T$, i.e. $(b, r_1)$.
\end{proof}



\begin{ex} \label{1.34}
  Use the preceding theorem to show that if $a = 51$ and $b = 15$, then $(51, 15) = (6, 3) = 3$.
\end{ex}

\begin{proof}
  Since $51 = 3 \cdot 15 + 6$, we find $(51, 15)$, we cite \ref{1.33} to see $(51, 15) = (15, 6)$. Then, since $15 = 2 \cdot 6 + 3$, we again cite \ref{1.33} to find $(15, 6) = (6, 3)$. We see that $(6, 3) = 3$ by inspection. Then, since equality is transitive, we conclude $(51, 15) = (6, 3) = 3$.
\end{proof}



\begin{ex} \label{1.35}
  Using the previous theorem and the Division Algorithm successively, devise a procedure for finding the greatest common divisor of two integers.
\end{ex}

Well you kind of gave the game away when you said to use \ref{1.33} and the divsion algorithm successively huh. If you're trying to find $(a, b)$, you simply invoke the division algorithm to get $a = nb + r$ (assuming WLOG that $a \geq b$), and then rewrite $(a, b)$ as $(b, r)$. Then, you use the divion algorithm to get $b = nr + r'$, simplifying to $(r, r')$, etc. etc., until at some point you have $(x, 0)$, which by inspection is equal to $x$.

You will always reach $(x, 0)$ because the divison algorithm produces a remainder $r$ that is strictly less than the smaller input $b$, so (informally) the smaller of the two numbers you're working with always gets smaller while never going negative.



\begin{ex} \label{1.36}
  Use the Euclidean Algorithm to find the following.
\end{ex}

\hspace*{5mm} \emph{1. $(96, 112)$} \\
\hspace*{15mm} $112 = 1 \cdot 96 + 16$, simplifying the problem to $(96, 16)$. Then $96 = 5 \cdot 16 + 0$, so we get $(16, 0) = 16$ \\

\hspace*{5mm} \emph{2. $(162, 31)$} \\
\hspace*{15mm} $162 = 5 \cdot 31 + 7 \implies (31, 7) \implies 31 = 4 \cdot 7 + 3 \implies (7, 3) \implies 7 = 2 \cdot 3 + 1 \implies (3, 1) = 1$. \\

\hspace*{5mm} \emph{3. $(0, 256)$} \\
\hspace*{15mm} Since everything divides $0$, this is trivially $256$. \\

\hspace*{5mm} \emph{4. $(-288,-166)$} \\
\hspace*{15mm} $-166 = 1 \cdot -288 + 122 \implies (-288, 122) \implies -288 = -3 \cdot 122 + 78 \implies (122, 78) \implies 122 = 1 \cdot 78 + 44 \implies (78, 44) \implies 78 = 1 \cdot 44 + 34 \implies (44, 34)
\implies 44 = 1 \cdot 34 + 10 \implies (34, 10) = 2$ by inspection.  \\

\hspace*{5mm} \emph{5. $(1, -2436)$} \\
\hspace*{15mm} Since the only integers that divide $1$ are $-1$, $0$, and $1$, we trivially find $1$. \\



\begin{ex} \label{1.37}
  Find integers $x$ and $y$ such that $162x + 31y = 1$.
\end{ex}

By division algorithm, $162 = 5 \cdot 31 + 7 \implies 7 = 1 \cdot 162 + (-5) \cdot 31$.

By division algorithm, $31 = 4 \cdot 7 + 3 \implies 3 = 1 \cdot 31 + (-4) \cdot 7 = 1 \cdot 31 + (-4) \cdot (1 \cdot 162 + (-5) \cdot 31) = (-4) \cdot 162 + 21 \cdot 31$.

By division algorithm, $7 = 2 \cdot 3 + 1 \implies 1 = 1 \cdot 7 + (-2) \cdot 3 = 1 \cdot (1 \cdot 162 + (-5) \cdot 31) + (-2) \cdot ((-4) \cdot 162 + 21 \cdot 31) = 9 \cdot 162 + (-47) \cdot 31$.

Thus, we've found our solution $x = 9$ and $y = -47$.



\pagebreak



\begin{thm} \label{1.38}
  Let $a$ and $b$ be integers. If $(a, b) = 1$, then there exist integers $x$ and $y$ such that $ax + by = 1$.
\end{thm}

\begin{proof}
  If either $a$ or $b$ is negative, replace it with $-a$ or $-b$ for the rest of this proof. At then end, you can replace either $x$ or $y$ with $-x$ or $-y$ to get an answer; for instance, if $a = -3$, we can replace $a = 3$, do the proof to obtain $x_0$ and $y_0$ such that $3x_0 + by_0 = 1$ and then realize that $(-3)(-x_0) + by_0 = 1$, which since $-x_0$ is still an integer still suffices. Now we will only be worrying about non-negative $a$ and $b$s.

  We will demonstrate an algorithm to find $x$ and $y$. WLOG, assume $a \geq b$. Invoke the division algorithm to get $a = n_1 b + r_1$. Then invoke it again to get $b = n_2 r_1 + r_2$. Then invoke it again to get $r_1 = n_3 r_2 + r_3$. Etc. etc. etc.

  We will show that the series ``remainder'' generated by this algorithm eventually has to hit $0$: in other words, $\exists i \in \mathbb{N} \ni r_i = 0$. To do this, we must notice that for any index $j$, since $r_j$ is generated by calling the division algorithm on $r_{j-2}$ and $r_{j-1}$, we find that $r_j \leq r_{j-1}-1$.
  Notice, then, that we can apply this to $r_{j-1}$ to obtain $r_{j-1} \leq r_{j-2} - 1$, and then plug that in to our previous inequality to get $r_{j} \leq r_{j-1} - 1 \leq r_{j-2} - 2$.

  By inspection (i.e. I'm lazy and don't want to formalize this), we notice we can continually apply this. We will apply this to $r_b$, and notice that $r_b \leq r_{b-1} - 1 \leq r_{b-2} - 2 \leq \cdots \leq r_{1} - (b - 1) \leq b - b$. Since $b-b = 0$, we find $r_b \leq 0$, but since $r_b$ is a remainder from the division algorithm we know $r_b \geq 0$, so we conclude $r_b = 0$.

  Notice we have \emph{not} proven that $r_b$ is the \emph{first} $0$, only that the remainders must \emph{eventually} reach $0$ at \emph{some} point.

  Now, keep invoking the division algorithm until the ``remainder'' generated by the algorithm is $0$: we will label that step $k+1$, so that we find $r_{k-1} = n_{k+1} r_k + 0$. We will show that $r_k$ is $1$.

  By invoking \ref{1.33} repeatedly, we find that $(a, b) = (b, r_1) = (r_1, r_2) = \cdots = (r_{k-1}, r_k) = (r_k, r_{k+1})$. Since $r_{k+1} = 0$, we conclude $(a, b) = (r_k, 0)$. Since $0$ divides everything, $(r_k, 0) = r_k$, so $(a, b) = r_k$, and since $a$ and $b$ are relatively prime we conclude $1 = r_k$.

  Now, we take all of our equations and rewrite them to solve for the remainder. For example, $a = n_1 b + r_1$ becomes $r_1 = a + (-n_1) b$, and $b = n_2 r_1 + r_2$ becomes $r_2 = b + (-n_2) r_1$.

  This gives us a bunch of equations of the form $r_j = \delta_{j} r_{j-2} + \gamma_{j} r_{j-1}$. This incldues one for $r_k$, namely $r_k = \delta_k r_{k-2} + \gamma_k r_{k-1}$.
  We can then substitute in lower indices of $r$ for $r_{k-2}$ and $r_{k-1}$, using the generic equation, to get something like $r_k = \delta_k (\delta_{k-2} r_{k-4} + \gamma_{k-2} r_{k-3}) + \gamma_k (\delta_{k-1} r_{k-3} + \gamma_{k-1} r_{k-2})$.

  That looks horrifying, but the important bit is that we notice if we simplify it we get $r_k = Ar_{k-4} + Br_{k-3} + Cr_{k-2}$ with $A, B, C \in \mathbb{Z}$. That is, \emph{by replacing all $r_j$'s with their respective equations, we have reduced the highest index on an $r$ in the right hand side by $1$}. Previously, the highest index was $k-1$, but now it's $k-2$, because we had an equation to represent $r_{k-1}$ in terms of $r_{k-3}$ and $r_{k-4}$.

  Notice, though, that not all $r$'s satisfy this property: namely, $r_1$ and $r_2$ simplify down to $a$ and $b$, which then don't have equations of their own. So, we apply the equations for $r_k$ through $r_1$ in ``reverse'' order, pairing down the maximum index of $k$ each time, until we're left with only $r_1$'s and $r_2$'s on the left hand side and can apply those equations to get a linear expression in $a$ and $b$ on the right hand side.

  We've been talking a lot about the right hand side, but remember, the left hand side is $r_k$, and we've shown $r_k = 1$, so we've just found a linear expression in $a$ and $b$ that is equal to $1$. In other words, $1 = ax + by$ for some $x, y \in \mathbb{Z}$.
\end{proof}



\begin{thm} \label{1.39}
  Let $a$ and $b$ be integers. If there exist integers $x$ and $y$ with $ax + by = 1$, then $(a, b) = 1$.
\end{thm}

\begin{proof}
  Readers of the last proof will be glad to hear this one is much simpler.

  By definition, $(a, b) | a$ and $(a, b) | b$. Then, $(a, b) | ax$ and $(a, b) | by$ by \ref{1.6}. Then, $(a, b) | ax + by$ by \ref{1.1}. Then, since $ax + by = 1$, we find $(a, b) | 1$. We know $1 | a$ and $1 | b$, so $(a, b) \geq 1$. The only number $\geq 1$ that divides $1$ is $1$, so since $(a, b) \geq 1$ and $(a, b) | 1$ we conclude $(a, b) = 1$.
\end{proof}



\begin{thm} \label{1.40}
  For any integers $a$ and $b$ not both $0$, there are integers $x$ and $y$ such that $ax + by = (a, b)$.
\end{thm}

\begin{proof}
  Let $c = a / (a, b)$ and $d = b / (a, b)$. Notice that since $(c, d) | c$ and $a = c \cdot (a, b)$ we find $\left( (c, d) \cdot (a, b) \right) | a$, and similarly since $(c, d) | d$ and $b = d \cdot (a, b)$ we find $\left( (c, d) \cdot (a, b) \right) | b$.

  Since $c$ and $d$ are integers not both $0$, $(c, d)$ must be a positive integer. Since $(c, d) \cdot (a, b)$ is a common factor of $a$ and $b$, and $(a, b)$ is the \emph{greatest} common factor of $a$ and $b$, we find $(c, d) \cdot (a, b) \leq (a, b) \implies (c, d) = 1$.

  Thus, we invoke \ref{1.38} to find integers $x$ and $y$ such that $cx + dy = 1$. Then, we multiply both sides by $(a, b)$ to find that $(a, b) \cdot cx + (a, b) \cdot dy = (a, b)$. Since $a = (a, b) \cdot c$ and $b = (a, b) \cdot d$ we conclude $ax + by = (a, b)$.
\end{proof}



\pagebreak



\begin{thm} \label{1.41}
  Let $a$, $b$, and $c$ be integers. If $a|bc$ and $(a, b) = 1$, then $a|c$.
\end{thm}

\begin{proof}
  Since $(a, b) = 1$, we can invoke \ref{1.38} to find $x, y \in \mathbb{Z} \ni ax + by = 1$.

  Now, since $a | bc$, we can cite \ref{1.6} to obtain $a | bcy$.

  Since $a \cdot 1 = a$ we find $a | a$, and then by \ref{1.6} we get $a | acx$.

  Then, by \ref{1.1} we get $a | (acx + bcy)$. We can then do some simple algebraic rearrangement to get $a | \left( c \cdot (ax + by) \right) \implies a | (c \cdot 1) \implies a | c$.
\end{proof}



\begin{thm} \label{1.42}
  Let $a$, $b$, and $n$ be integers. If $a|n$, $b|n$, and $(a, b) = 1$, then $ab | n$
\end{thm}

\begin{proof}
  Since $a | n$ and $b | n$ we find integers $k, j$ such that $ak = n$ and $bj = n$. By the transitive property of equality, $ak = bj$. Since $j$ is an integer, we conclude $b | ak$. Since $(a, b) = 1$, we invoke \ref{1.41} to find $b | k$. Thus, we invoke an integer $d$ such that $bd = k$. Substiuting this into $ak = n$, we find $abd = n$, and since $d$ is an integer we conclude $ab | n$.
\end{proof}



\begin{thm} \label{1.43}
  Let $a$, $b$, and $n$ be integers. If $(a,n) = 1$ and $(b, n) = 1$, then $(ab, n) = 1$.
\end{thm}

\begin{proof}
  Invoking \ref{1.38} twice, we find two pairs of integers, $x_a, y_a, x_b,$ and $y_b$ such that $ax_a + ny_a = 1$ and $bx_b + ny_b = 1$. We notice then that $(ax_a + ny_a) \cdot (bx_b + ny_b) = 1 \cdot 1 = 1$, and we simplify the left-hand side to $ax_abx_b + ax_any_b + ny_abx_b + ny_any_b = ab(x_a x_b) + n(ax_a y_b + y_a b x_b + ny_a y_b) = 1$, and then by closure of the integers and \ref{1.39} we find that $(ab, n) = 1$.
\end{proof}



\begin{ques} \label{1.44}
  What hypotheses about $a$, $b$, $c$, and $n$ could be added so that $ac \equiv bc \gmod n$? State an appropriate theorem and prove it before reading on.
\end{ques}

\newtheorem*{IKFMS}{I know this from Math Seminar}
\begin{IKFMS}
  $ac \equiv bc \gmod n$ implies $a \equiv b \gmod n$ if and only if $(c, n) = 1$.
\end{IKFMS}

\begin{proof}
  First, we will show that $ac \equiv bc \gmod n$ and $(c, n) = 1$ imply that $a \equiv b \gmod n$.

  Notice that $ac \equiv bc \gmod n$ implies $n | (bc - ac)$. By distribution, we obtain $n | \left( c (b - a) \right)$. Then, since $(c, n) = 1$, we cite \ref{1.41} to obtain $n | (b - a)$, which by definition means $a \equiv b \gmod n$.

  Now, we will show that if $(c, n) > 1$, then $ac \equiv bc \gmod n$ does \emph{not} imply $a \equiv b \gmod n$.

  We will do this by example. Notice that $n | \left( c \cdot (n / (c, n)) \right)$: the right-hand side can be rearranged to read $n \cdot (c / (c, n))$ and $c / (c, n)$ is an integer because $(c, n)$ is a factor of $c$. Then, since $n|n$, we cite \ref{1.6} to find $n | \left( n \cdot (c / (c, n)) \right)$.
  We then cite the facts that $n | 0$ and \ref{1.2} to find $n | \big( (c \cdot (n / (c, n))) - 0 \big)$, and we can substitue in $c \cdot 0$ for $0$ to find $n | \big( (c \cdot (n / (c, n))) - c \cdot 0 \big)$. We then, by definition, obtain $c \cdot (n / (c, n)) \equiv c \cdot 0 \gmod n$.

  \emph{However}, since $n > 0$ (because congruence ``modulo $n$'' is defined) and $(c, n) > 1$, we find that $0 < n / (c, n) < n$. This implies that $n / (c, n) \not\equiv 0 \gmod n$, despite teh fact that $c \cdot (n / (c, n)) \equiv c \cdot 0 \gmod n$, giving us our counterexample.
\end{proof}



\begin{thm} \label{1.45}
  Let $a$, $b$, $c$, and $n$ be integers with $n > 0$. If $ac \equiv bc \gmod n$ and $(c, n) = 1$, then $a \equiv b \gmod n$.
\end{thm}

See \ref{1.44}.



\begin{ques} \label{1.46}
  Suppose $a$, $b$, and $c$ are integers and that there is a solution to the linear Diophantine equation $ax + by = c$. That is, suppose there are integer $x$ and $y$ that satisfy the equation $ax + by = c$. What condition must $c$ satisfy in terms of $a$ and $b$?
\end{ques}

Since $(a, b) | (ax + by)$, we conclude $(a, b) | c$.



\begin{ques} \label{1.47}
  Can you make a conjecture by completing the following statement?
\end{ques}

\begin{PC} \label{PC 1.47}
  Given integers $a$, $b$, and $c$, there exist integers $x$ and $y$ that satisfy the equation $ax + by = c$ if and only if $(a, b) | c$.
\end{PC}

\begin{proof}
  Notice that an integer solution to $ax + by = c$ implies that, since $(a, b) | a$ and $(a, b) | b \implies (a, b) | (ax + by)$ (\ref{1.6} and \ref{1.1}), we conclude $(a, b) | c$.

  Now, notice $(a, b) | c \implies \exists d \in \mathbb{Z} \ni d(a, b) = c$. We invoke \ref{1.40} to find integers $w$ and $z$ such that $aw + bz = (a, b)$. Then, we can multiply both sides by $d$ to obtain $d (aw + bz) = d (a,b)$, which simplifies to $awd + bzd = c$, giving us the solution $x = wd$ and $y = zd$.
\end{proof}



\begin{thm} \label{1.48}
  Given integers $a$, $b$, and $c$ with $a$ and $b$ not both $0$, there exist integers $x$ and $y$ that satisfy the equation $ax + by = c$ if and only if $(a, b) | c$.
\end{thm}

See Paul's Conjecture \ref{PC 1.47}.



\pagebreak



\begin{ques} \label{1.49}
  For integers $a$, $b$, and $c$, consider teh linear Diophantine equation $ax + by = c$. Suppose integers $x_0$ and $y_0$ satisfy the equation: that is, $ax_0 + by_0 = c$. What other values
  $$x = x_0 + h \mbox{ and } y = y_0 + k$$
  also satisfy $ax + by = c?$ Formulate a conjecture that answers this question. Devise some numerical examples to ground your exploration. For example, $6(-3) + 15 \cdot 2 = 12$. Can you find other integers $x$ and $y$ such that $6x + 15y = 12$? How many other pairs of integers $x$ and $y$ can you find? Can you find infintely many other solutions?
\end{ques}

\begin{PC} \label{PC 1.49}
  The integers $x_1 = x_0 + h$ and $y_1 = y_0 + k$ satisfy the equation $ax_1 + by_1 = c$ if and only if $\frac{b}{(a, b)} | h$ and $k = - \frac{ah}{b})$.
\end{PC}

\begin{proof}
  First, notice $ax_1 + by_1 = c$ if and only if $a(x_0 + h) + b(y_0 + k) = c$. Then with rearrangement, we find this is equivalent to $ax_0 + by_0 + ah + bk = c \iff c + ah + bk = c \iff ah + bk = 0$. Then, we find $bk = -ah \iff k = -(ah/b)$.

  Notice that this ``if-and-only-if chain'' doesn't show that $k$ is an \emph{integer}. Thus, we will show that $k$ is an integer if and only if $(b / (a, b)) | h$, the other condition, to complete our proof.

  First, notice $\frac{b}{(a,b)} | h \implies \exists d \in \mathbb{Z} \ni \frac{b}{(a, b)} d = h$. Then, $\frac{b}{(a, b)} da = ah$. This can be rewritten as $b \cdot (d \frac{a}{(a,b)}) = ah$, and since $a / (a,b)$ is an integer we conclude $b | ah$. In other words, $k = - (ah / b)$ is an integer.

  Going the opposite direction is much the same: $k = - (ah / b)$ being an integer implies $b | ah$, implying $bd = ah$, implying $bd / (a, b) = ah / (a, b)$, implying $\frac{b}{(a,b)} | \frac{ah}{(a,b)}$.
  Then, we notice that since there exist integers $\gamma$ and $\delta$ such that $a\gamma + b\delta = (a,b)$ (\ref{1.40}), we find $\frac{a}{(a,b)} \gamma + \frac{b}{(a,b)} \delta = \frac{(a,b)}{(a,b)} = 1$, which by \ref{1.39} implies that $(\frac{a}{(a,b)}, \frac{b}{(a,b)}) = 1$.
  Thus, we cite \ref{1.41} with $\frac{b}{(a,b)} | \frac{ah}{(a,b)}$ to find $\frac{b}{(a,b)} | h$.
\end{proof}



\begin{ex} \label{1.50}
  A farmer lays out the sum of $1,770$ crowns in purchasing horses and oxen. He pays $31$ crowns for each horse and $21$ crowns for each ox. What are the possible numbers of horses and oxen that the farmer bought?
\end{ex}

$51$ horses and $9$ oxen is the first situation I found. Using Paul's Conjecture \ref{PC 1.49}, we can find that further solutions can be found by subtracting $21$ from the number of horses while adding $31$ to the number of oxen (trust me it makes sense).

$30$ horses and $40$ oxen.

$9$ horses and $71$ oxen.



\begin{thm} \label{1.51}
  Let $a$, $b$, $c$, $x_0$, and $y_0$ be integers with $a$ and $b$ not both $0$ such that $ax_0 + by_0 = c$. Then the integers
  $$x = x_0 + \frac{b}{(a, b)} \mbox{ and } y = y_0 - \frac{a}{(a,b)}$$
  also satisfy the linear Diophantine equation $ax + by = c$.
\end{thm}

\begin{proof}
  Notice these integers satisfy the requirements for Paul's Conjecture \ref{PC 1.49} (I'm too lazy to show how but 1.53 will force me to).
\end{proof}



\begin{thm} \label{1.52}
  If $a$, $b$, and $c$ are integers with $a$ and $b$ not both $0$, and the linear diophantine equation $ax + by = c$ has at least one integer solution, can you find a general expression for all the integer solutions to that equation? Prove your conjecture.
\end{thm}

\begin{PC} \label{PC 1.52}
  The set of all pairs of integers $(x_1, y_1)$ such that $ax_1 + by_1 = c$ can be written as $$\left\{ \left(x_0 + \frac{bd}{(a,b)}, y_0 - \frac{ad}{(a,b)} \right) \mid d \in \mathbb{Z} \right\}$$
\end{PC}

\begin{proof}
Paul's Conjecture \ref{PC 1.49} can easily be extended here: if we let the integer solution given be $x_0$ and $y_0$, such that $ax_0 + by_0 = c$, we want to find a general expression for all integers $x_1 = x_0 + h$ and $y_1 = y_0 + k$ where $\frac{b}{(a,b)} | h$ and $k = - \frac{ah}{b}$.

The set of all integers $h$ such that $\frac{b}{(a,b)} | h$ can be expressed as $\{\frac{bd}{(a,b)} \mid d \in \mathbb{Z}\}$. The corresponding $k$ value for any $h$ is $- \frac{ah}{b} = - \frac{a (bd / (a,b)}{b} = - \frac{ad}{(a,b)}$.
Thus, any pair of $x_1 = x_0 + \frac{bd}{(a, b)}$ and $y_1 = y_0 - \frac{ad}{(a,b)}$ satisfies the Diophantine equation $ax_1 + by_1 = c$.
\end{proof}



\begin{thm} \label{1.53}
  Let $a$, $b$, and $c$ be integers with $a$ and $b$ not both $0$. If $x = x_0$, $y = y_0$ is an integer solution to the equation $ax + by = c$ (that is, $ax_0 + by_0 = c$) then for every integer $k$, the numbers
  $$x = x_0 + \frac{kb}{(a,b)} \mbox{ and } y = y_0 - \frac{ka}{(a,b)}$$
  are integers that also satisfy the linear Diophantine equation $ax + by = c$. Moreover, every solution to the linear Diophantine equation $ax + by = c$ is of this form.
\end{thm}

\begin{proof}
  This is just a less pretentious way of saying Paul's Conjecture \ref{PC 1.52} that doesn't involve set notation.
\end{proof}



\end{document}
