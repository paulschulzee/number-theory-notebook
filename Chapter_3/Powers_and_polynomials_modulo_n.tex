% !TEX root = ../main.tex

\documentclass[../main.tex]{subfiles}

\begin{document}

\begin{ex} \label{3.1}
  Show that $41$ divides $2^{20} - 1$ by following these steps. Explain why each step is true.
\end{ex}

1. $2^5 \equiv -9 \gmod{41}$ because $41 | 41 = 32 - (-9)$

2. $(2^5)^4 \equiv (-9)^4 \gmod{41}$ because \ref{1.14}

3. $2^{20} \equiv 81^2 \gmod{41}$ because $(2^5)^4 = 2^{5 \cdot 4} = 2^{20}$ and $(-9)^4 = ((-9)^2)^2 = 81^2$

4. $2^{20} - 1 \equiv 0 \gmod{41}$ because $81 \equiv -1 \gmod{41}$ so $81^2 \equiv 1 \gmod{41}$ so $2^{20} \equiv 1 \gmod{41}$



\begin{ques} \label{3.2}
  In your head, can you find the natural number $k$, $0 \leq k \leq 11$, such that $k \equiv 37^{453} \gmod{12}$?
\end{ques}

$1$



\begin{ques} \label{3.3}
  In your head or using paper and pencil, but no calculator, can you find the natural number $k$, $0 \leq k \leq 6$, such that $2^{50} \equiv k \gmod{7}$?
\end{ques}

$4$



\begin{ques} \label{3.4}
  Using paper and pencil, but no calculator, can you find the natural number $k$, $0 \leq k \leq 11$, such that $39^{453} \equiv k \gmod{12}$?
\end{ques}

$3$



\begin{ex} \label{3.5}
  Show that $39$ divides $17^{48} - 5^{24}$
\end{ex}

So open up python and... no? Fine.

Notice (via calculator) that $17^2 \equiv 16 \gmod{39}$. Then, $17^3 \equiv 16 \cdot 17 \equiv -1 \gmod{39}$. From there we obtain $17^4 \equiv -17$ and $17^5 \equiv -16$ before arriving at $17^6 \equiv 1$. This cycle then repeats: we notice that $48$ is a multiple of $6$, so $17^{48} \equiv 1 \gmod{39}$.

Similarly, $5^4 \equiv 1 \gmod{39}$ so we find $5^{24} \equiv 1$. Then we know $17^{48} \equiv 5^{24} \gmod{39}$, so definitionally $39 | (17^{48} - 5^{24})$.



\begin{ques} \label{3.6}
  Let $a$, $n$, and $r$ be natural numbers. Describe how to find the number $k$ ($0 \leq k \leq n-1$) such that $k \equiv a^{r} \mod n$ subject to the restraint that you never multiply numbers larger than $n$ and that you only have to do about $\log_2 (r)$ such multiplications.
\end{ques}

Get a computer to do it.

Failing that, or if you're the poor soul who has to program the computer, you start off by essentially making a library of powers of $a$ in their ``simplest form'' modulo $n$. So you start with $a \equiv x_0 \gmod n$ (where in most cases if $a < n$ we find $x_0 = a$, but then you square $a$ and then pair down the result to something between $0$ and $n$ (probably using the division algorithm) to get $a^2 \equiv x_1$, then square that to get $a^4 \equiv x_2$, $a^8 \equiv x_3$, etc. etc. until you've build up to $x_{\lfloor \log_2 (r) \rfloor}$. Then add together the relevant terms from largest to smallest (e.g. $a^{39} \equiv a^{32} + a^4 + a^2 + a \equiv x_5 + x_2 + x_1 + x_0$).



\setcounter{thm}{13}

\begin{thm} \label{3.14}
  Given any integer $a$ and any natural number $n$, there exists a unique integer $t$ in the set $\{0, 1, 2, \ldots, n-1\}$ such that $a \equiv t \gmod n$.
\end{thm}

\begin{proof}
  By division algorithm, we find $a = qn + r$ with $0 \leq r \leq n-1$. Notice $qn = a-r$ implies $n | (a-r)$ implies $a \equiv r \gmod n$ with $0 \leq r \leq n-1$, so the existence of such a $t$ is shown.

  Say there exist $t_1,t_2$ in the set $\{0,\ldots,n-1\}$ such that $a \equiv t_1 \gmod n$ and $a \equiv t_2 \gmod n$. Then $t_1 \equiv t_2 \gmod n$ by \ref{1.11}. Then $n | (t_1 - t_2)$.
  However, since $0 \leq t_1, t_2 \leq n-1$, we find $-n+1 \leq t_1 - t_2 \leq n-1$. The only number that $n$ divides in that range is $0$, so we conclude $t_1 - t_2 = 0$ or $t_1 = t_2$, proving that such a $t$ is unique.
\end{proof}



\begin{ex} \label{3.15}
  Find three complete residue systems modulo $4$: the canoncial complete residue system, one containing negative numbers, and one containing no two consecutive numbers.
\end{ex}

CCRS (canonical complete residue system): $\{0, 1, 2, 3\}$.

CRSCNN (complete residue system containing negative numbers): $\{-4, -3, -2, -1\}$.

CRSCNTCN (complete residue system containing no two consecutive numbers): $\{0, 5, 2, 7\}$.



\pagebreak



\begin{thm} \label{3.16}
  Let $n$ be natural number. Every complete residue system modulo $n$ contains $n$ elements.
\end{thm}

\begin{proof}
  Well we know that the quotient group $\mathbb{Z} / n\mathbb{Z}$ has... no? Fine.

  If a CRS has more than $n$ elements, by invoking \ref{3.14} repeatedly we find by pigeonhole principle there have to be two elements of the CRS (call them $a$ and $b$) that are equivalent to the same member of the set $\{0, \ldots, n-1\}$. By \ref{1.11}, we then find $a \equiv b \gmod n$. Then it's impossible for every integer to be congruent to exactly one member of the set, because we find $a \equiv a \gmod n$ and $a \equiv b \gmod n$, so our CRS is not a CRS at all. Thus, all CRS's will have $n$ or fewer elements.

  If a CRS has fewer than $n$ elements, if we invoke \ref{3.14} on all of its elements it won't ``fully cover'' the set $\{0, \ldots, n-1\}$. Let $x$ be the uncovered element. It's impossible for $x$ to be congruent to any of the elements of the CRS, since by \ref{3.14} all of the elements of the CRS are congruent to \emph{exactly} one element of $\{0, \ldots, n-1\}$ and so they can't be congruent to both the element we ``covered'' earlier and to $x$. Thus, our CRS is not a CRS at all, and we conclude all CRS's will have $n$ or more elements.

  Since the size of any CRS is greater than $n-1$ and less than $n+1$, we conclude that all CRS's are of size $n$.
\end{proof}



\begin{thm} \label{3.17}
  Let $n$ be a natural number. Any set, $A = \{a_1, a_2, \ldots, a_n\}$ of $n$ integers for which no two are congruent modulo $n$ is a complete residue set.
\end{thm}

\begin{proof}
  If we invoke \ref{3.14} on all of the members of $A$ we will get exactly the elements of $S = \{0, \ldots, n-1\}$ (we cannot hit any element twice because then we would have two members of $A$ that are congruent by \ref{1.11}, and we cannot hit any element outside of $S$ because of \ref{3.14}). Then by \ref{3.14} and \ref{1.11}, we notice that every integer is congruent to exactly one element in $S$, and thus it must be congruent to its corresponding element in $A$ and only that element of $A$ (or else it would be congruent to two elements of $S$, too).
\end{proof}

\end{document}
