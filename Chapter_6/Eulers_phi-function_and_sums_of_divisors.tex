% !TEX root = ../main.tex

\documentclass[../main.tex]{subfiles}

\begin{document}


\begin{ex} \label{6.10}
  Compute each of the following sums
\end{ex}

1. $\sum_{d | 6} \phi(d) = 1 + 1 + 2 + 2 = 6$

2. $\sum_{d|10} \phi(d) = 1 + 1 + 4 + 4 = 10$

3. $\sum_{d|24} \phi(d) = 1 + 1 + 2 + 2 + 2 + 4 + 4 + 8 = 24$

4. $\sum_{d|36} \phi(d) = 1 + 1 + 2 + 2 + 2 + 6 + 4 + 6 + 12 = 36$

5. $\sum_{d|27} \phi(d) = 1 + 2 + 6 + 18 = 27$

\emph{Make a sweeping conjecture about the sum of $\phi(d)$ taken over all natural divisors of any natural number n.}

\begin{PC} \label{PC 6.18}
  For any natural number $n$,

  $$\sum_{d | n} \phi(d) = n$$
\end{PC}



\begin{thm} \label{6.11}
  If $p$ is a prime, then

  $$\sum_{d|p} \phi(d) = p$$
\end{thm}

\begin{proof}
  $\sum_{d|p} \phi(d) = \phi(1) + \phi(p) = 1 + (p-1) = p$.
\end{proof}



\begin{thm} \label{6.12}
  If $p$ is a prime, then

  $$\sum_{d|p^k} \phi(d) = p^k$$
\end{thm}

\begin{proof}
  First, notice $\phi(p^i) = p^i - p^{i-1}$ for any natural $i$. This is because there are $p^i$ natural numbers less than or equal to $p^i$, and 1 in every $p$ of them is a multiple of $p$ and thus not relatively prime (i.e. $p^{i} / p = p^{i-1}$ of them are multiple of $p$).

  $\sum_{d|p^k} \phi(d) = \sum_{i=0}^k \phi(p^i) = 1 + \sum_{i=1}^k p^i - p^{i-1} = p^k$.
\end{proof}



\begin{thm} \label{6.13}
  If $p$ and $q$ are two different primes, then

  $$\sum_{d|pq} \phi(d) = pq$$
\end{thm}

\begin{proof}
  There are $pq$ natural numbers less than or equal to $pq$. One in $p$ are a multiple of $p$, one in $q$ are a multiple of $q$, and the rest are relatively prime to $pq$. Thus $\phi(pq) = pq - q - p + 1$ (the $+1$ is because we subtracted out $pq$ twice, once as a multiple of $p$ and once as a multiple of $q$.

  $\sum_{d|pq} \phi(d) = \phi(1) + \phi(p) + \phi(q) + \phi(pq) = 1 + (p-1) + (q-1) + (pq - q - p + 1) = pq$
\end{proof}



\begin{thm} \label{6.14}
  If $n$ and $m$ are relatively prime natural numbers, then

  $$\left( \sum_{d|m} \phi(d) \right) \cdot \left( \sum_{d|n} \phi(d) \right) = \sum_{d|mn} \phi(d)$$
\end{thm}

\begin{proof}
  $(\sum_{d|m} \phi(d)) \cdot (\sum_{d|n} \phi(d)) = \sum_{d_1|m} \sum_{d_2|n} \phi(d_1) \cdot \phi(d_2)$ by distribution. By \ref{2.31}, for every possible $d_1$ and $d_2$ we know $(d_1, d_2) = 1$. Thus, we can simplify the above as $\sum_{d_1|m, d_2 | n} \phi(d_1 \cdot d_2)$.

  Since $n$ and $m$ are relatively prime, every factor $d$ of $mn$ is a unique combination of some $d_1 | m$ and $d_2 | n$, and vice versa. In other words, the set of factors of $mn$ is the same as the set $\{d_1 \cdot d_2 : d_1 | m, d_2 | n\}$, by \ref{2.12} (trust me). Thus, we can simplify further to find $(\sum_{d|m} \phi(d)) \cdot (\sum_{d|n} \phi(d)) = \sum_{d|mn} \phi(d)$.
\end{proof}



\begin{thm} \label{6.15}
  If $n$ is a natural number, then

  $$\sum_{d|n} \phi(d) = n$$
\end{thm}

\begin{proof}
  By FTA, any $n = p_1^{r_1} \cdot p_2^{r_2} \cdot \cdots \cdot p_m^{r_m}$ for distinct primes $p_1$ through $p_m$ and natural numbers $r_1$ through $r_m$. We will induct on $m$.

  Our base case is $m = 0$, which is \ref{6.12}.

  Our inductive hypothesis is that for all $k < m$, if $n = p_1^{r_1} \cdot \cdots \cdot p_k^{r_k}$ then $\sum_{d|n} \phi(d) = n$. Our inductive step must show that this holds for $m$.

  Say $n = p_1^{r_1} \cdot \cdots \cdot p_{m-1}^{r_{m-1}} \cdot p_m^{r_m}$. Let $\eta = p_1^{r_1} \cdot \cdots \cdot p_{m-1}^{r_{m-1}}$. Notice by our inductive hypothesis that $\sum_{d|\eta} \phi(d) = \eta$ and $\sum_{d|p_m^{r_m}} \phi(d) = p_m^{r_m}$.
  Also notice by \ref{2.12} that $(\eta, p_m^{r_m}) = 1$. Then, by \ref{6.14} we see $n = \eta \cdot p_m^{r_m} = \left( \sum_{d|\eta} \phi(d) \right) \cdot \left( \sum_{d|p_m^{r_m}} \phi(d) \right) = \sum_{d|\eta \cdot p_m^{r_m}} \phi(d) = \sum_{d|n} \phi(d)$.

  Removing the middle steps, we find $n = \sum_{d|n} \phi(d)$.
\end{proof}



\begin{ex} \label{6.16}
  For a natrual number $n$ consider the fractions

  $$\frac{1}{n}, \frac{2}{n}, \frac{3}{n}, \ldots, \frac{n}{n},$$

  all written in reduced form. Try to find a natural one-tone correspondence between the reduced fractions and the number $\phi(d)$ for $d | n$. Show how that ovservation provides a very clever proof to the preceding theorem.
\end{ex}

\begin{proof}
  For any $d | n$, there are $\phi(d)$ fractions that have $d$ as their denominator in simplest form.

  Notice that for any $d | n$, the list of fractions above will contain all of $\frac{1}{d}, \frac{2}{d}, \ldots, \frac{d}{d}$. This is because for any $u \leq d$, we know $\frac{u}{d} \cdot (n/d) = \frac{u(n/d)}{n}$, and $u(n/d) = n(u/d) \leq n$ because $u/d \leq 1$.

  Thus, there are $d$ ``candidates'' for fractions with $d$ as their denominator in simplest form. Notice that since simplest form is when the numerator and denominator are relatively prime, these fractions do have $d$ as their denominator in simplest form exactly when the numerator is relatively prime to $d$. Since the numerator can be any value from $1$ to $d$, this means the number of fractions with $d$ as the denominator in simplest form is equal to the number of natruals form $1$ to $d$ that are relatively prime to $d$, or $\phi(d)$.

  Notice, then, that every fraction in the series $\frac{1}{n}, \ldots, \frac{n}{n}$ has to have a simplest form (with some divisor of $n$ as the denominator). Thus, if we add up the number of fractions with $d$ as their denominator in simplest form for all $d | n$, we'll get the number of fractions (which is $n$, since there are $n$ possibilities for the numerator). In other words, since $\phi(d)$ is the number of fractions with $d$ in their denominator in simplest form, we're saying $n = \sum_{d | n} \phi(d)$.
\end{proof}



\begin{thm} \label{6.17}
  Every prime $p$ has $\phi(p-1)$ primitive roots.
\end{thm}

\begin{proof}
  It's the last week of high school, this is going to be pretty informal.

  There are $p-1$ numbers less than $p$ that all have to have some order. Since we know for any $k$ that $k^{p-1} \equiv 1 \gmod p$ by \ref{4.15} (Fermat's Little), we know by \ref{4.10} that the order of every number needs to divide $p-1$. Thus, the possible orders are the divisors of $p-1$. For each divisor $d$, there are $\phi(d)$ ``slots'' (as only $\phi(d)$ incongruent numbers can have order $d$ by \ref{6.5}).
  Thus, the total number of ``slots'' is $\sum_{d|p-1} \phi(d)$, which by \ref{6.15} means there are $p-1$ slots. Since there are $p-1$ numbers, this means every slot must be filled: there are $\phi(p-1)$ ``slots'' for numbers with order $p-1$, and so since all are filled there must be $\phi(p-1)$ numbers with order $p-1$, i.e. $\phi(p-1)$ primitive roots of $p$.
\end{proof}


\end{document}
