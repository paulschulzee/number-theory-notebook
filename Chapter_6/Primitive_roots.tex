% !TEX root = ../main.tex

\documentclass[../main.tex]{subfiles}

\begin{document}


\begin{thm} \label{6.4}
  Suppose $p$ is a prime and $\ord_p (a) = d$. Then for each natural number $i$ with $(i, d) = 1$, $\ord_p(a^i) = d$.
\end{thm}

\begin{proof}
  To find $k = \ord_p (a^i)$, we're looking for the smallest number such that $(a^i)^k \equiv 0 \gmod p$, or in other words $a^{ik} \equiv 0 \gmod p$. By \ref{4.10}, this is only the case when $d | ik$. Since $(i, d) = 1$, we know $d | ik$ implies $d | k$ (easily shown by \ref{2.12}. I swear to God we proved this somewhere, but this is like the 5th time I've gone to cite this imaginary theorem and I never find it). The smallest (natural) $k$ such that $d | k$ is $d$, so $k = \ord_p (a^i) = d$.
\end{proof}



\pagebreak



\begin{thm} \label{6.5}
  For a prime $p$ and a natural number $d$, at most $\phi (d)$ incongruent integers modulo $p$ have order $d$ modulo $p$.
\end{thm}

\begin{proof}
  Either there is an integer with order $d$, or there isn't. If there isn't we're done, so let's assume there is one and call it $a$.

  We will show that all numbers with order $d$ are congruent to an integer of the form $a^x$ where $(d, x) = 1$ and $x \leq d$. As there are only $\phi(d)$ possible values of $x$, this will mean that all integers with order $d$ will be congruent to one of $\phi(x)$ integers, and thus there can at most be $\phi(x)$ incongruent integers with order $d$.

  Let $k$ be an integer with order $d$. Let $A = \{a, a^2, a^3, \ldots, a^d\}$ and $K = \{a, a^2, a^3, \ldots, a^d\}$. By \ref{4.8}, we know both of these sets are pairwise incongruent. We also know that every element of $A$ is a solution to the equation $x^d - 1 \equiv 0 \gmod p$, as $(a^i)^d - 1 \equiv (a^d)^i - 1 \equiv 1^i - 1 \equiv 0 \gmod p$. Similarly for $K$.
  By \ref{6.3}, there can only be $d$ incongruent solutions to this equation. Thus, since $A$ is full of $d$ pairwise incongruent integers that are all solutions, it must contain one number congruent to each solution. Same with $K$. Thus, for any element of $A$, there must be a congruent element in $K$, and vice versa. Since $k \in K$, we know $k = a^i$ for some $i < d$.

  To see that $(d, i) = 1$, we notice that $k^{d/(d,i)} \equiv a^{d \cdot \frac{i}{(d,i)}} \equiv 1 \gmod p$ (\ref{4.10}) Since the order of $k$ is $d$, we then know $d/(d,i) \geq d$, implying $1 \geq (d,i)$, which means $(d,i) = 1$.
\end{proof}



\begin{thm} \label{6.6}
  Let $p$ be a prime and suppose $g$ is a primitive root modulo $p$. Then the set $\{0, g, g^2, g^3, \ldots, g^{p-1}\}$ forms a CRS modulo $p$.
\end{thm}

\begin{proof}
  Since this set has $p$ elements, by \ref{3.17}, all we have to do to show this is a CRS is to show that the terms are pairwise incongruent modulo $p$.

  None of the powers will ever be congruent to $0$, because that would imply $p | g^n$ which is impossible since no $p$ can appear in the prime factorization of $g^n$ (as if it did, $g \equiv 0 \gmod p$ and then $g$ has no order).

  Say there are two integers $i, j$ such that $1 \leq i, j \leq p-1$ and $g^i \equiv g^j \gmod p$. Assume WLOG $i > j$. Then $g^i \equiv g^j \gmod p$ implies $g^{i-j} \cdot g^j \equiv 1 \cdot g^j \gmod p$, which by \ref{4.5} means $g^{i-j} \equiv 1 \gmod p$.
  By \ref{5.10}, this means $(p-1) | (i - j)$, but since $0 \leq i - j \leq p-2$ the only multiple of $p-1$ that the difference could be is $0$, implying $i = j$. Thus, if two of the nonzero members of the set are congruent, they are the same element. In other words, they are pairwise incongruent modulo $p$.
\end{proof}



\begin{ex} \label{6.7}
  For each of the primes $p$ less than $20$ find a primitive root and make a chart showing what powers of the primitive root give each of the natural numbers less than $p$.
\end{ex}

$2$: $1^1 \equiv 1$.

$3$: $2^2 \equiv 1$, $2^1 \equiv 2$.

$5$: $3^4 \equiv 1$, $3^3 \equiv 2$, $3^1 \equiv 3$, $3^2 \equiv 4$.

$7$: $3^6 \equiv 1$, $3^2 \equiv 2$, $3^1 \equiv 3$, $3^4 \equiv 4$, $3^5 \equiv 5$, $3^3 \equiv 6$.

$11$: $2^{10} \equiv 1$, $2^1 \equiv 2$, $2^8 \equiv 3$, $2^2 \equiv 4$, $2^4 \equiv 5$, $2^9 \equiv 6$, $2^7 \equiv 7$, $2^3 \equiv 8$,
$2^6 \equiv 9$, $2^5 \equiv 10$.

$13$: $2^{12} \equiv 1$, $2^1 \equiv 2$, $2^4 \equiv 3$, $2^2 \equiv 4$, $2^9 \equiv 5$, $2^5 \equiv 6$, $2^{11} \equiv 7$, $2^3 \equiv 8$,
$2^8 \equiv 9$, $2^{10} \equiv 10$, $2^7 \equiv 11$, $2^6 \equiv 12$.

$17$: $3^{16} \equiv 1$, $3^{11} \equiv 2$, $3^1 \equiv 3$, $3^{12} \equiv 4$, $3^5 \equiv 5$, $3^{15} \equiv 6$, $3^{11} \equiv 7$, $3^{10} \equiv 8$,
$3^2 \equiv 9$, $3^3 \equiv 10$, $3^7 \equiv 11$, $3^{13} \equiv 12$, $3^4 \equiv 13$, $3^9 \equiv 14$, $3^6 \equiv 15$, $3^8 \equiv 16$.

$19$: $2^{18} \equiv 1$, $2^1 \equiv 2$, $2^{13} \equiv 3$, $2^2 \equiv 4$, $2^{16} \equiv 5$, $2^{14} \equiv 6$, $2^6 \equiv 7$, $2^3 \equiv 8$,
$2^8 \equiv 9$, $2^{17} \equiv 10$, $2^{12} \equiv 11$, $2^{15} \equiv 12$, $2^5 \equiv 13$, $2^7 \equiv 14$, $2^{11} \equiv 15$, $2^4 \equiv 16$,
$2^{10} \equiv 17$, $2^9 \equiv 18$.



\begin{thm} \label{6.8}
  Every prime $p$ has a primitive root.
\end{thm}

\begin{proof}
  It says we'll come back to this one.
\end{proof}



\begin{ex} \label{6.9}
  Consider the prime $p = 13$. For each divisor $d = 1, 2, 3, 4, 6, 12$ of $12 = p-1$, mark which of the natural numbers in the set $\{1, 2, 3, \ldots, 12\}$ have order $d$.
\end{ex}

Order $1$: Just $1$.

Order $2$: Just $12$.

Order $3$: $3$, $9$.

Order $4$: $5$, $8$.

Order $6$: $4$, $10$.

Order $12$: $2$, $6$, $7$, $11$.


\end{document}
