% !TEX root = ../main.tex

\documentclass[../main.tex]{subfiles}

\begin{document}


\begin{thm} \label{6.1}
  Let $f(x) = a_n x^n + a_{n-1} x^{n-1} + \cdots + a_0$ be a polynomial of degree $n > 0$ with integer coefficients and assume $a_n \neq 0$. Then an integer $r$ is a root of $f(x)$ if and only if htere exists a polynomial $g(x)$ of degree $n-1$ with integer coefficients such that $f(x) = (x-r)g(x)$.
\end{thm}

\begin{proof}
  To see that the existence of $g(x)$ such that $f(x) = (x-r) g(x)$ implies $r$ is a root is easy: notice $f(r) = (r-r) g(r) = 0 \cdot g(r) = 0$.

  The other direction requires some polynomial long division, which I won't Tex out because that would be a.) hell for me and b.) uninformative for the reader. Instead I will assert without evidence that $f(x) / (x-r)$ simplifies out to $a_n x^{n-1} + (a_n r + a_{n-1})x^{n-2} + \cdots + (a_n r^{n-1} + a_{n-1} r^{n-2} + \cdots + a_0) + a_n r^n + a_{n-1}r^{n-1} + \cdots + a_0$, with the part from $a_n r^n$ on being the remainder. Notice, then, that since $f(r) = 0$, the remainder is $0$. Looking at the coefficients on the rest of the terms, we notice they're all made up of integers added and multiplied together, and thus must all be integers. Thus, $f(x) / (x-r)$ is a satisfactory $g(x)$ such that $f(x) = (x-r) g(x)$.
\end{proof}



\begin{thm} \label{6.2}
  Let $f(x) = a_n x^n + a_{n-1} x^{n-1} + \cdots + a_0$ be a polynomial of degree $n > 0$ with integer coefficients and $a_n \neq 0$. Let $p$ be a prime number and $r$ an integer. Then, if $f(r) \equiv 0 \gmod p$, there exists a polynomial $g(x)$ of degree $n-1$ such that

  $$(x-r)g(x) = a_nx^n + a_{n-1}x^{n-1} + \cdots + a_1x + b_0$$

  where $a_0 \equiv b_0 \gmod p$
\end{thm}

\begin{proof}
  We divide $f(x) / (x-r)$ as above, but instead of noticing the remainder is $0$ we notice the remainder is congruent to $0$ modulo $p$. If we let the quotient (without the remainder) be $g(x)$, then we notice $f(x) = (x-r) g(x) + R$, where $R$ is the remainder. Rearranging, we obtain $(x-r) g(x) = f(x) - R = a_n x^n + \cdots + a_1 x + a_0 - R$.

  From here, we combine $a_0 - R$ into one term $b_0$, noticing that since $p | R$ we know $a_0 \equiv b_0 \gmod p$, and obtain $(x-r) g(x) = a_n x^n + \cdots + a_1 x + b_0$, and we're done.
\end{proof}



\begin{thm} \label{6.3}
  If $p$ is a prime and $f(x) = a_n x^n + a_{n-1} x^{n-1} + \cdots + a_0$ is a polynomial with integer coefficients and $a_n \neq 0 \gmod p$, then $f(x) \equiv 0 \gmod p$ has at most $n$ non-congruent solutions modulo $p$.
\end{thm}

\begin{proof}
  We will induct on $n$. Our base case is $n = 0$, in which we are trying to find solutions to $a_0 \equiv 0 \gmod p$ where $a_0 \not \equiv 0 \gmod p$. Obviously there are $0$ solutions to this.

  Our induction hypothesis is that for $m < n$, any polynomial $g(x) = a_m x^m + a_{m-1} x^{m-1} + \cdots + a_0$ has at most $m$ non-congruent modulo $p$ solutions to the equation $g(x) \equiv 0 \gmod p$. Our induction step hopes to show this property holds for $n$.

  Say $f(x) = a_n x^n + a_{n-1} x^{n-1} + \cdots a_0$. If the equation $f(x) \equiv 0 \gmod p$ has no solutions, we're done, so we'll go ahead and assume it has one solution $x = r$. Then by \ref{6.2} we find $f(x) \equiv (x-r) g(x)$ for some $g(x)$ of order less than $n$.

  Notice that any solution $k$ to $f(x) \equiv 0 \gmod p$ finds $f(k) \equiv (k-r) \cdot g(k) \equiv 0 \gmod p$. In other words, $p | (k-r)g(k)$. By \ref{2.27}, either $p | (k-r)$ (i.e. $k \equiv r \gmod p$) or $p | g(k)$ (i.e. $g(k) \equiv 0 \gmod p$).
  Thus, \emph{any solution to $f(x) \equiv 0 \gmod p$ that is incongruent to $r$ modulo $p$ must also be a solution to $g(x) \equiv 0 \gmod p$}. By our induction hypothesis, we know there are strictly less than $n$ pairwise incongruent modulo $p$ solutions to $g(x) \equiv 0 \gmod p$, so adding in $r$ means we know there are \emph{no more than} $n$ pairwise incongruent modulo $p$ solutions to  $f(x) \equiv 0 \gmod p$.
\end{proof}


\end{document}
