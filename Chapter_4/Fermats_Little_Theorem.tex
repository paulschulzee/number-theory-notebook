% !TEX root = ../main.tex

\documentclass[../main.tex]{subfiles}

\begin{document}

\begin{ques} \label{4.7}
  Choose some relatively prime natural numbers $a$ and $n$ and compute the order of $a$ modulo $n$. Frame a conjecture concerning how large the order of $a$ modulo $n$ can be, depending on $n$.
\end{ques}

We covered this in math seminar right? Am I going crazy? Maybe AMR? I'm unsure.

We know $\ord_n (a)$ divides the number of units of $n$, where the units of $n$ are the numbers in the CCRS that have a multiplicative inverse modulo $n$.



\begin{thm} \label{4.8}
  Let $a$ and $n$ be natural numbers with $(a, n) = 1$ and let $k = \ord_n (a)$. Then the numbers $a^1, a^2, \ldots, a^k$ are pariwise incongruent modulo $n$.
\end{thm}

\begin{proof}
  Say there exist $1 \leq i, j \leq k$ such that $a^i \equiv a^j \gmod n$. WLOG, assume $j > i$. Then, notice that $n | (a^j - a^i)$ implies $n | (a^i \cdot (a^{j-i} - 1))$, which since $(n, a^i) = 1$ (\ref{4.2}) implies $n | (a^{j-i} - 1)$, implying $a^{j-i} \equiv 1 \gmod n$ for some $j-1 < k$.

  This is a contradiction, since $k$ is the \emph{smallest} integer such that $a^k \equiv 1$. Thus, our assumption that such $i$ and $j$ exist is flawed, and the numbers $a^1, a^2, \ldots, a^k$ are pariwise incongruent.
\end{proof}



\begin{thm} \label{4.9}
  Let $a$ and $n$ be natural numbers with $(a, n) = 1$ and let $k = \ord_n (a)$. For any natural number $m$, $a^m$ is congruent modulo $n$ to one of the numbers $a^1, a^2, \ldots, a^k$.
\end{thm}

\begin{proof}
  By division algorithm, $m = kq + r$ for some integers $q, r$ with $r < k$. Then, we find $a^m \equiv a^{kq + r} \equiv a^{kq} \cdot a^r \equiv (a^k)^q \cdot a^r \equiv 1^q \cdot a^r \equiv a^r \gmod n$ for some $0 \leq r \leq k-1$. Our one edge case is $a^m \equiv a^0 \equiv 1$, but we can clearly see this implies $a^m \equiv a^k$, and we're done.
\end{proof}



\pagebreak



\begin{thm} \label{4.10}
  Let $a$ and $n$ be natural numbers with $(a, n) = 1$, let $k = \ord_n (a)$, and let $m$ be a natural number. Then $a^m \equiv 1 \gmod n$ if and only if $k | m$.
\end{thm}

\begin{proof}
  Seeing that $k | m$ implies $a^m \equiv 1 \gmod n$ is easy: $a^m \equiv a^{kx} \equiv (a^k)^x \equiv 1^x \equiv 1 \gmod n$, for some integer $x$.

  To show $a^m \equiv 1 \gmod n$ implies $k | m$, we invoke the division algorithm and find $q, r$ such that $m = kq + r$ for some $0 \leq r < k$. Notice $1 \equiv a^m \equiv a^{kq + r} \equiv a^{kq} \cdot a^r \equiv (a^k)^q \cdot a^r \equiv 1^q \cdot a^r \equiv a^r \gmod n$.
  Since $a^r \equiv 1 \gmod n$, we conclude $r$ isn't between $1$ and $k-1$ inclusive, as then $k$ wouldn't be the \emph{smallest} number such that $a^k \equiv 1 \gmod n$. Thus, since $0 \leq r < k$, the only remaining possibility is that $r = 0$. Thus, $m = kq + r$ becomes $m = kq + 0 = kq$, implying $k | m$.
\end{proof}



\begin{thm} \label{4.11}
  Let $a$ and $n$ be natural numbers with $(a, n) = 1$. Then $\ord_n (a) < n$ (unless $n = 1$).
\end{thm}

\begin{proof}
  Suppose $\ord_n (a) \geq n$. Then, by \ref{4.8}, we know $a^1, a^2, \ldots, a^n$ are pairwise incongruent modulo $n$. This implies that each of these numbers reduces to a different number in the CCRS modulo $n$ (if two numbers reduced to the same thing, they'd be congruent). Thus, since the CCRS has $n$ elements like the sequence, every member of the CCRS must be ``hit'' by the powers of $a$. This implies there exists an $i$ such that $a^i \equiv 0 \gmod n$, or in other words $n | a^i$.
  Since $(a, n) = 1$, by \ref{4.2} we know $(a^i, n) = 1$ Since $n$ is a common factor of $n$ and $a^i$, we conclude $n = 1$. This is the one exception to the theorem: in any other case, we now know $\ord_n (a) < n$.
\end{proof}



\begin{ex} \label{4.12}
  Compute $a^{p-1} \gmod p$ for various numbers $a$ and primes $p$, and make a conjecture.
\end{ex}

$a = 4$, $p = 5$, $4^{5-1} \equiv 1 \gmod 5$.

$a = 3$, $p = 5$, $3^{5-1} \equiv 1 \gmod 5$.

$a = 4$, $p = 7$, $4^{7-1} \equiv 1 \gmod 7$.

$a = 3$, $p = 7$, $3^{7-1} \equiv 1 \gmod 7$.

It seems like $a^{p-1} \equiv 1 \gmod p$.



\begin{thm} \label{4.13}
  Let $p$ be a prime and let $a$ be an integer not divisible by $p$; that is, $(a, p) = 1$. Then $A = \{a, 2a, 3a, \ldots, pa\}$ is a complete residue system modulo $p$.
\end{thm}

\begin{proof}
  By \ref{3.17}, we only have to show that no two members of $A$ are congruent modulo $p$. To do this, take any two members $ia$ and $ja$, and assume WLOG that $j > i$. Since $0 < j - i < p$, we know $p \not | \; (j-i)$, and we also know $p \not | \; a$. Thus, by \ref{2.27}, we know $p \not | \; (j-i)a$, or in other words $p \not | \; (aj - ai)$.
  By definition, this implies $aj \not \equiv ai \gmod p$, so the members of $A$ are pairwise incongruent and this (along with the fact that $A$ has $p$ elements) implies $A$ is a compelte residue system.
\end{proof}



\begin{thm} \label{4.14}
  Let $p$ be a prime and let $a$ be an integer not divisible by $p$. Then

  $$a \cdot 2a \cdot 3a \cdot \cdots \cdot (p-1)a \equiv 1 \cdot 2 \cdot 3 \cdot \cdots \cdot (p-1) \gmod p$$
\end{thm}

\begin{proof}
  By \ref{4.13}, we know the set $\{a, 2a, \ldots, pa\}$ is a CRS. Because of this, when we take the members of this set and ``reduce'' them to their CCRS members, we should exactly cover the CCRS. Now we kick out $pa$, since we know $pa \equiv 0 \gmod p$ (since $p | pa$). Goodbye, $pa$. What we're left with is a set that has one term equivalent to $1$ modulo $p$, one term equivalent to $2$ modulo $p$, etc., all the way up to a term equivalent to $p-1$ modulo $p$.

  Now, we take the product $a \cdot \cdots \cdot (p-1)a$ and replace each term with its corresponding meber of the CCRS. This won't change what it's congruent to mod $p$, since we're replacing things that are equivalent mod $p$. What we'll be left with, then, is $1 \cdot 2 \cdot \cdots \cdot p-1$, albeit probably not in that order.
\end{proof}



\begin{thm} \label{4.15}
  If $p$ is a prime and $a$ is an integer relatively prime to $p$, then $a^{p-1} \equiv 1 \gmod p$.
\end{thm}

\begin{proof}
  By \ref{4.14}, we know $a \cdot 2a \cdot \cdots \cdot (p-1)a \equiv 1 \cdot 2 \cdot \cdots \cdot (p-1) \gmod p$. With some simple rearrangement, we see $a^{p-1} \cdot (1 \cdot \cdots \cdot (p-1)) \equiv 1 \cdot (1 \cdot \cdots \cdot (p-1))$.
  Notice, then, that by \ref{2.12} and the fact that $p$ is prime we know that $1 \cdot \cdots \cdot (p-1)$ is relatively prime to $p$. By \ref{4.5}, we conclude $a^{p-1} \equiv 1 \gmod p$.
\end{proof}



\begin{thm} \label{4.16}
  If $p$ is a prime and $a$ is any integer, then $a^p \equiv a \gmod p$.
\end{thm}

\begin{proof}
  See \ref{4.17} and \ref{4.15}.
\end{proof}



\pagebreak



\begin{thm} \label{4.17}
  The two versions of Fermat's Little Theorem above are equivalent.
\end{thm}

\begin{proof}
  Showing \ref{4.16} implies \ref{4.15} is as simple as realizing $a^p \equiv a \gmod p$ implies $a \cdot a^{p-1} \equiv a \cdot 1 \gmod p$ and invoking \ref{4.5}.

  Showing \ref{4.15} implies \ref{4.16} is as simples as multiply both sides of $a^{p-1} \equiv 1 \gmod p$ by $a$... for $a$ relatively prime to $p$. If $a$ is not relatively prime to $p$, that implies $p | a$, so $a^p \equiv 0 \equiv a \gmod p$, an easy edge case.
\end{proof}



\begin{thm} \label{4.18}
  Let $p$ be a prime and $a$ be an integer. If $(a, p) = 1$, then $\ord_p (a)$ divides $p-1$, that is, $\ord_p(a) | p-1$.
\end{thm}

\begin{proof}
  Since $a^{p-1} \equiv 1 \gmod p$ by \ref{4.15}, we cite \ref{4.10} and are done.
\end{proof}



\begin{ex} \label{4.19}
  Compute each of the followin without the aid of a calculator or computer.
\end{ex}

1. $512^{372} \equiv (512^{12})^{31} \equiv 1^{31} \equiv 1 \gmod{13}$.

2. $3444^{3233} \equiv (10^{16})^{202} \cdot 10^1 \equiv 1^{202} \cdot 10 \equiv 10 \gmod{17}$.

3. $123^{456} \equiv (8^{22})^{20} \cdot 8^16 \equiv 1^{20} \cdot 18^8 \equiv 2^4 \equiv 4^2 \equiv 8 \gmod{23}$.



\begin{ex} \label{4.20}
  Find the remainder upon division of $314^{159}$ by $31$
\end{ex}

$314^{159} \equiv (4^{30})^5 \cdot 4^9 \equiv 1^5 \cdot 4^9 \equiv 4 \cdot 16^4 \equiv 4 \cdot 8^2 \equiv 4 \cdot 2 \equiv 8 \gmod{31}$



\begin{thm} \label{4.21}
  Let $n$ and $m$ be natural numbers that are relatively prime, and let $a$ be an integer. If $x \equiv a \gmod n$ and $x \equiv a \gmod m$, then $x \equiv a \gmod{nm}$
\end{thm}

\begin{proof}
  $x \equiv a \gmod n$ means $n | (x-a)$. $x \equiv a \gmod m$ means $m | (x-a)$. Since $(n, m) = 1$, we cite \ref{2.25} to get $nm | (x-a)$, which means $x \equiv a \gmod{nm}$.
\end{proof}



\begin{ex} \label{4.22}
  Find the reaminder when $4^{72}$ is divided by $91$ ($= 7 \cdot 13)$.
\end{ex}

$4^{72} \equiv 2^{36} \equiv 4^{18} \equiv 2^9 \equiv 2 \cdot 4^4 \equiv 2 \cdot 2^2 \equiv 1 \gmod 7$.

$4^{72} \equiv 3^{36} \equiv 9^{18} \equiv 3^9 \equiv 3 \cdot 9^4 \equiv 3 \cdot 3^2 \equiv 1 \gmod{13}$.

By \ref{4.21}, $4^{72} \equiv 1 \gmod{91}$.



\begin{ex} \label{4.23}
  Find the natural number $k < 117$ such that $2^{117} \equiv k \gmod{117}$.
\end{ex}

$2^{117} \equiv 2 \cdot 4^{49} \equiv 2 \cdot 4 \cdot 16^{24} \equiv 8 \cdot 22^{12} \equiv 8 \cdot 16^6 \equiv 8 \cdot 22^3 \equiv 8 \cdot 22 \cdot 16 \equiv 59 \cdot 16 \equiv 8 \gmod{117}$

\end{document}
